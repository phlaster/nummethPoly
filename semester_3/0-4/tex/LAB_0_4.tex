\documentclass[a4paper]{article}
\usepackage[russian]{babel}
\usepackage[utf8]{inputenc}
\usepackage{setspace}
\setstretch{1.5}
\usepackage{anyfontsize}
\usepackage{amsmath, amsfonts, amssymb, amsthm, mathtools}
\usepackage{geometry}
\usepackage{graphicx}
\usepackage{wrapfig}
\usepackage{xcolor}
\usepackage{multicol}
\usepackage{listings}
\geometry{a4paper, portrait, margin=10mm, bmargin=15mm, tmargin=15mm}

%%%%%%%%%%%%%%%%%%%%%%%%%%%%%%%%%%%%%%%%%%%%%%%%%%%%%%%%%%%%%%%%%%%%%%%%%%%%
%%%%%%%%%%%%%%%%%%%%%%%%%%%%%%%%%%%%%%%%%%%%%%%%%%%%%%%%%%%%%%%%%%%%%%%%%%%%
%%%%%%%%%%%%%%%%%%%%%%%%%%%%%%%%%%%%%%%%%%%%%%%%%%%%%%%%%%%%%%%%%%%%%%%%%%%%

\begin{document}

\include{titlepage}

%%%%%%%%%%%%%%%%%%%%%%%%%%%%%%%%%%%%%%%%%%%%%%%%%%%%%%%%%%%%%%%%%%%%%%%%%%%%
%%%%%%%%%%%%%%%%%%%%%%%%%%%%%%%%%%%%%%%%%%%%%%%%%%%%%%%%%%%%%%%%%%%%%%%%%%%%
%%%%%%%%%%%%%%%%%%%%%%%%%%%%%%%%%%%%%%%%%%%%%%%%%%%%%%%%%%%%%%%%%%%%%%%%%%%%

\newpage
\pagenumbering{arabic}
\setcounter{page}{2}
{\Large{\textbf{1. Создать случайную СЛАУ заданной размерности с заданным числом обусловленности}}}\\
Зададим случайный вектор-столбец длины $\mathbf{4}$:
\begin{align}
    \mathbf{w}_0 =
    \begin{pmatrix}
        0.680 \\
        0.032 \\
        0.585 \\
        0.610
    \end{pmatrix}
\end{align}
Нормируем $\mathbf{w_0}$ в соответствии с евклидовой нормой:
\begin{align}
    \mathbf{w} = \cfrac{\mathbf{w}_0}{||\mathbf{w}_0||_2} =
    \frac{
    \begin{pmatrix}
        0.680 \\
        0.032 \\
        0.585 \\
        0.610
    \end{pmatrix}}{\sqrt{0.680^2+0.032^2+0.585^2+0.610^2}} =
    \begin{pmatrix}
        0.627 \\
        0.030 \\
        0.539 \\
        0.562
    \end{pmatrix}
\end{align}
Солгасно формуле преобразования Хаусхолдера
\begin{align}
    \mathbf{Q = E - 2ww}^T
\end{align}
получим матрицу $\mathbf{Q}$:
\begin{align}
    \mathbf{Q} =
    \begin{pmatrix}
        1 & 0 & 0 & 0 \\
        0 & 1 & 0 & 0 \\
        0 & 0 & 1 & 0 \\
        0 & 0 & 0 & 1
    \end{pmatrix} - 2
    \begin{pmatrix}
        0.627 \\
        0.030 \\
        0.539 \\
        0.562
    \end{pmatrix} \cdot
    \begin{pmatrix}
        0.627 \\
        0.030 \\
        0.539 \\
        0.562
    \end{pmatrix} ^T =
    \begin{pmatrix}
        0.214 & -0.037 &  -0.676 &  -0.704 \\
        -0.037 &   0.998 & -0.032 & -0.033 \\
        -0.676 & -0.032  &  0.419 &  -0.606 \\
        -0.704 & -0.033  & -0.606  &   0.369
    \end{pmatrix}
\end{align}
Её число обусловленности заведомо известно и равно длине $\mathbf{w}$:
\begin{align}
    \mathtt{cond(Q) == 4}
\end{align}

{\Large{\textbf{2. Решить СЛАУ методом LU-разложения с выбором главного элемента}}}\\
Зададим случайный вектор неизвестных $\mathbf{x}$ и вычислим вектор правой части $\mathbf{b}$ уравнения $\mathbf{Qx}=\mathbf{b}$:
\begin{align}
    \begin{pmatrix}
        0.214 & -0.037 &  -0.676 &  -0.704 \\
        -0.037 &   0.998 & -0.032 & -0.033 \\
        -0.676 & -0.032  &  0.419 &  -0.606 \\
        -0.704 & -0.033  & -0.606  &   0.369
    \end{pmatrix} \cdot
    \begin{pmatrix}
        0.132 \\
        0.443 \\
        0.488 \\
        0.957
    \end{pmatrix} =
    \begin{pmatrix}
        -0.992 \\
         0.390 \\
        -0.478 \\
        -0.050
    \end{pmatrix}
\end{align}
LU-разложение с выбором главного (опорного) элемента это т.н. LUP-разложение, где исходная невырожденная (квадратная с ненулевым определителем) матрица $\mathbf{Q}$ представляется в виде:
\begin{align}
    \mathbf{PQ} = \mathbf{LU}
\end{align}
Здесь матрица $\mathbf{L}$ -- нижнетреугольная с единицами на главной диагонали, $\mathbf{U}$ -- верхнетреугольная общего вида, а $\mathbf{P}$ -- «матрица перестановок», получаемая из единичной матрицы путём перестановки строк или столбцов. $\mathbf{P}$ соответствует вектору перестановок $\mathbf{p}$, в соответствии с которым нужно поменять местами строки $\mathbf{LU}$, чтобы получить $\mathbf{Q}$.\\

{\large\bf{I. Aлгоритм LUP-разложения:}}\\
0. Инициализируем вектор $\mathbf{p}_0$ числами натурального ряда:
\vspace{-0.2cm}
\[
\mathbf{p}_0 =
\begin{pmatrix}
    1 \\
    2 \\
    3 \\
    4
\end{pmatrix}
\]
1. В первом столбце $\mathbf{M}_0 = \mathbf{Q}$ найдём элемент с наибольшим модулем:
\[
    \mathbf{M}_0 =
    \begin{pmatrix}
        0.214 & -0.037 &  -0.676 &  -0.704 \\
        -0.037 &   0.998 & -0.032 & -0.033 \\
        -0.676 & -0.032  &  0.419 &  -0.606 \\
        \mathbf{-0.704} & -0.033  & -0.606  &   0.369
    \end{pmatrix}
\]
2.  Переставим строку с опорным элементом так, чтобы он оказался на главной диагонали, отразив соответствующую перестановку в векторе перестановок:
\[
    \mathbf{M}_1 =
    \begin{pmatrix}
        \mathbf{-0.704} & \mathbf{-0.033}  & \mathbf{-0.606}  & \mathbf{0.369}\\
        -0.037 &   0.998 & -0.032 & -0.033 \\
        -0.676 & -0.032  &  0.419 &  -0.606 \\
        \mathit{0.214} & \mathit{-0.037} &  \mathit{-0.676} &  \mathit{-0.704}
    \end{pmatrix}; \hspace{1cm}
        \mathbf{p}_1 =
    \begin{pmatrix}
        \mathbf{4} \\
        2 \\
        3 \\
        \mathit{1}
    \end{pmatrix}
\]
3. Все элементы первого столбца лежащие под опорным элементом (ниже 1-й строки) разделим на него:
\[
\mathbf{M}_2 =
\begin{pmatrix}
    \mathbf{-0.704} & -0.033  & -0.606  & 0.369\\
    \mathit{\frac{-0.037}{-0.704}} &   0.998 & -0.0319 & -0.033 \\
    \mathit{\frac{-0.676}{-0.704}} & -0.032  &  0.419 &  -0.606 \\
    \mathit{\frac{0.214}{-0.704}} & -0.037 & -0.676 &  -0.704
\end{pmatrix} =
\begin{pmatrix}
    \mathbf{-0.704} & -0.033  & -0.606  & 0.369\\
    \mathit{0.053} &   0.998 & -0.0319 & -0.033 \\
    \mathit{0.960} & -0.032  &  0.419 &  -0.606 \\
    \mathit{-0.304} & -0.037 & -0.676 &  -0.704
\end{pmatrix}
\]
4. Далее из элементов каждого следующего столбца, лежащих ниже опорной (1-й) строки вычтем произведение соответствующего элемента первого столбца и первой строки:
\[
\mathbf{M}_3 =
\begin{pmatrix}
    -0.704 & \mathbf{-0.033}  & \mathbf{-0.606}  & \mathbf{0.369}\\
    \mathit{0.053} &   0.998-(-0.033\cdot0.053) & -0.0319-(-0.606\cdot0.053) & -0.033-(0.369\cdot0.053) \\
    \mathit{0.960} & -0.032-(-0.033\cdot0.960)  &  0.419-(-0.606\cdot0.960) &  -0.606-(0.369\cdot0.960) \\
    \mathit{-0.304} & -0.037-(-0.033\cdot(-0.304)) & -0.676-(-0.606\cdot(-0.304)) &  -0.704-(0.369\cdot(-0.304))
\end{pmatrix}=
\]
\[
=
\begin{pmatrix}
    -0.704 & -0.033 & -0.606 &  0.369 \\
     0.053 &  1     &  0     & -0.053 \\
     0.959 &  0     &  1     & -0.959\\
    -0.304 & -0.047 & -0.860 & -0.592 \\
\end{pmatrix}
\]
5. Далее переходим ко второму столбцу, повторяя пп. 1-4. В нашем случае, элемент с наибольшим модулем уже находится на главной диагонали, так что сразу переходим к шагу вычитания:
\[
\mathbf{M}_4 =
\begin{pmatrix}
    -0.704 & -0.033 & -0.606 &  0.369 \\
    0.053 &  1     &  \mathbf{0}     & \mathbf{-0.053} \\
    0.959 &  \mathit{0}     &  1-(0\cdot0)     & -0.959-(-0.053\cdot0)\\
    -0.304 & \mathit{-0.047} & -0.860-(-0.047\cdot0) & -0.592-(-0.047\cdot(-0.053)) \\
\end{pmatrix} =
\begin{pmatrix}
    -0.704 & -0.033 & -0.606 &  0.369 \\
    0.053 &  1     &  0     & -0.053 \\
    0.959 &  0     &  1     & -0.959\\
    -0.304 & -0.047 & -0.860 & -0.595 \\
\end{pmatrix}
\]
6. Теперь к третьему столбцу (также, сразу шаг вычитания):
\[
\mathbf{M}_5 =
\begin{pmatrix}
    -0.704 & -0.033 & -0.606 &  0.369 \\
    0.053 &  1     &  0     & -0.053 \\
    0.959 &  0     &  1     & \mathbf{-0.959}\\
    -0.304 & -0.047 & \mathit{-0.860} & -0.595-(-0.959\cdot(-0.860)) \\
\end{pmatrix} =
\begin{pmatrix}
    -0.704 & -0.033 & -0.606 &  0.369 \\
    0.053 &  1     &  0     & -0.053 \\
    0.959 &  0     &  1     & -0.959\\
    -0.304 & -0.047 & -0.860 & -1.420 \\
\end{pmatrix}
\]
7. По четвёртому (последнему) столбцу никаких операций проводить не нужно, т.к. элемент с наибольшим модулем уже оказался на главной диагонали, а других элементов ниже и правее него в матрице не осталось. Полученная же матрица $\mathbf{M}_5=\mathbf{M}$ вкупе с вектором перестановок $\mathbf{p}_1=\mathbf{p}$ являются компактным представлением LUP-разложения матрицы $\mathbf{Q}$. По-отдельности матрицы $\mathbf{L}$ и $\mathbf{U}$ можно получить пользуясь соотношением:
\begin{align}
    \mathbf{M} = \mathbf{L} - \mathbf{E} + \mathbf{U}
\end{align}
То есть:
\begin{align}
\mathbf{L} =
\begin{pmatrix}
    1    &  0    &  0    &  0\\
    0.053&  1    &  0    &  0\\
    0.959&  0    &  1    &  0\\
   -0.304& -0.047& -0.860&  1
\end{pmatrix}; \hspace{0.5cm}
\mathbf{U} =
\begin{pmatrix}
   -0.704& -0.033& -0.606&  0.369\\
    0    &  1    &  0    & -0.053\\
    0    &  0    &  1    & -0.959\\
    0    &  0    &  0    & -1.420
\end{pmatrix}; \hspace{0.5cm}
\mathbf{p} =
\begin{pmatrix}
    4 \\ 2 \\ 3 \\ 1
\end{pmatrix}
\end{align}

{\large\bf II. Решение СЛАУ с помощью полученного разложения:}\\
1. Прямая подстановка -- решение уравнения $\mathbf{Ly} = \mathbf{Pb}=\mathbf{b}_p$ относительно $\mathbf{y}$:
\[
(\mathbf{L}|\mathbf{b}_p) =
\left(
\begin{array}{cccc|c}
    1    &  0    &  0    & 0 &-0.050 \\
    0.053&  1    &  0    & 0 & 0.390 \\
    0.959&  0    &  1    & 0 &-0.478 \\
   -0.304& -0.047& -0.860& 1 &-0.992
\end{array}
\right) \implies
\mathbf{y} =
\begin{pmatrix}
    -0.050 \\
     0.392 \\
    -0.430 \\
    -1.359
\end{pmatrix};
\]
2. Обратная подстановка -- решение уравнения $\mathbf{U}\widetilde{\mathbf{x}} = \mathbf{y}$
\[
(\mathbf{U}|\widetilde{\mathbf{x}}) =
\left(
\begin{array}{cccc|c}
    -0.704& -0.033& -0.606& 0.369& -0.050\\
    0    &  1    &  0    & -0.053&  0.392 \\
    0    &  0    &  1    & -0.959& -0.430\\
    0    &  0    &  0    & -1.420& -1.359
\end{array}
\right) \implies
\widetilde{\mathbf{x}} =
\begin{pmatrix}
   0.132\\
   0.443\\
   0.488\\
   0.957
\end{pmatrix};
\]
3. Рассчитаем вектор невязки по формуле:
\begin{align}
    \mathbf{r}=\mathbf{b}-\textbf{Q}\widetilde{\mathbf{x}}
\end{align}
\[
\mathbf{r}=
\begin{pmatrix}
    -0.992 \\
     0.390 \\
    -0.478 \\
    -0.050
\end{pmatrix} -
\begin{pmatrix}
     0.214 & -0.037 & -0.676 & -0.704 \\
    -0.037 &  0.998 & -0.032 & -0.033 \\
    -0.676 & -0.032 &  0.419 & -0.606 \\
    -0.704 & -0.033 & -0.606 &  0.369
\end{pmatrix} \cdot
\begin{pmatrix}
    0.132\\
    0.443\\
    0.488\\
    0.957
\end{pmatrix} =
\begin{pmatrix}
   -11.102\\
    -5.551\\
    -5.551\\
    0
\end{pmatrix} \cdot 10^{-17}
\]
\[
||\mathbf{r}||_2=1.360\cdot10^{-16}
\]

{\Large{\textbf{3. Воспользоваться запрограммированным методом и операцией Matlab «$\backslash$» на тестовых матрицах с нулевым определителем}}}\\

{\large\bf I. Результат работы алгоритма на C++:}
{\singlespacing
\begin{verbatim}
    -> Проверка матрциы A: <-
1     2     3
4     5     6
7     8     9
    -> Расчитать число обусловленности не удалось!
    -> LU-разложение:
L:              |  U:
1      0     0  |  7    8     9
0.571  1     0  |  0    0.857 1.714
0.143  0.5   1  |  0    0     1.110e-16
    -> Вектор перестановок:
[2, 0, 1]
    -> Вычислим вектор неизвестных, основываясь на LU-разложении:
[6.33828e+16, -1.26766e+17, 6.33828e+16]
    -> Вектор невязки найденного решения:
[-3.89447, 55.5759, -3.92036]
    -> 2-норма вектора невязки:
55.85

    -> Проверка матрциы B: <-
1e+08   2e+08   3e+08
4e+08   5e+08   6e+08
7e+08   8e+08   9e+08
    -> Число обусловленности:
2.83006e+17
    -> Для решения системы нужно произвести LU-разложение матрицы:
     ! Невозможно диагонализовать матрицу! Нулевой элемент на диагонали. Пропуск шага.
    -> LU-разложение:
L:               |  U:
1       0     0  |  7e+08  8e+08     9e+08
0.571   1     0  |  0      8.571e+07 1.714e+08
0.143   0.5   1  |  0      0         0
    -> Вектор перестановок:
[2, 0, 1]
    -> Вычислим вектор неизвестных, основываясь на LU-разложении:
[-nan, -inf, inf]
    -> Вектор невязки найденного решения:
[-nan, -nan, -nan]
    -> 2-норма вектора невязки:
-nan
\end{verbatim}}

\newpage
{\large\bf II. Результат работы программы Matlab:}
\begin{multicols}{2}
{\singlespacing
\begin{verbatim}
Warning: Matrix is close to singular or badly scaled.
> In LAB_0_4__3 (line 19)
1     2     3
4     5     6
7     8     9
Det(A): 6.6613e-16
Cond(A): 1.143944118188076e+17
||rA||: 4.3504
Warning: Matrix is singular to working precision.
> In LAB_0_4__3 (line 26)
100000000   200000000   300000000
400000000   500000000   600000000
700000000   800000000   900000000
Det(B): 0
Cond(B): 2.556914355281162e+16
||rB||: NaN
\end{verbatim}}
\end{multicols}
{\large\bf III. Комментарий:}
\begin{multicols}{2}
Различия в результатах обусловлены различием во внутренней реализации алгоритмов решения СЛАУ. В самодельном алгоритме на C++ расчёт числа обусловленности проходит через вычисление определителя наивным рекурсивным алгоритмом с факториальной сложностью, а расчёт обратной матрицы невозможен, если определитель в точности равен нулю. В случае с матрицей $\mathbf{B}$ в расчётах нарастает неточность, связанная с округлениями чисел с плавающей точкой, поэтому определитель всё же отличен от нуля. В реализации же пакета Matlab определители считаются, скорее всего, эффективным алгоритмом перемножения диагональных элементов матрциы $\mathbf{U}$ из $\mathbf{LU}$-разложения. В обеих случаях решения, предоставленные программами формально неверны, т.к. определители обоих матриц должны быть равны нулю в точности, что означает неопределённость чисел обусловленности и наличие нетривиальных решений СЛАУ (или их полное отсутствие).
\end{multicols}
{\Large{\textbf{4. В MatLab Решить СЛАУ с матрицами Гильберта}}}\\
Матрицы Гильберта -- особый класс квадратных матриц, использующийся для проверки алгоритмов и оценки численной стабильности. Они представляют собой модель системы линейных уравнений с плохой обусловленностью. Элементы матрицы задаются формулой:
\begin{align}
    H_{ij}={\frac {1}{i+j-1}},i,j=1,2,3,...,n
\end{align}
Ожидается, что решение прямыми методами СЛАУ с матрицами Гильберта будет иметь большую погрешность из-за их плохой обусловленности. Результат выполнения скрипта:
{\singlespacing\begin{verbatim}
Размер матрицы: 5
Число обусловленности: 476607.2502
Норма фактической ошибки: 3.0881e-13
Норма невязки: 1.2469e-13

Размер матрицы: 10
Число обусловленности: 16024909625167.58
Норма фактической ошибки: 0.00013871
Норма невязки: 3.7693e-05

Размер матрицы: 15
Число обусловленности: 3.378778714747708e+17
Норма фактической ошибки: 3.298
Норма невязки: 1.0649
\end{verbatim}}
Налицо стремительный рост числа обусловленности и нормы невязки решения для случайного вектора правой части по причинам, упомянутым выше, что согласуется с ожиданиями.
\newpage
{\Large{\textbf{5. Построить график временных затрат решения СЛАУ для операции Matlab «$\backslash$» в зависимости от размера матрицы}}}\\
Рис.1 График зависимости времени решения СЛАУ от размера случайной матрицы. Число обусловленности = 10.\\
\includegraphics[width=1\linewidth]{../pics/task_5}
Рис.2 Подгонка кривой степенной зависимости в интерактивном режиме.\\
\includegraphics[width=1\linewidth]{../pics/cfit}








\end{document}
