\documentclass[a4paper]{article}
\usepackage[russian]{babel}
\usepackage[utf8]{inputenc}
\usepackage{setspace}
\setstretch{1.5}
\usepackage{anyfontsize}
\usepackage{amsmath, amsfonts, amssymb, amsthm, mathtools}
\usepackage{geometry}
\usepackage{graphicx}
\usepackage{wrapfig}
\usepackage{xcolor}
\usepackage{multicol}
\usepackage{listings}
\geometry{a4paper, portrait, margin=10mm, bmargin=15mm, tmargin=15mm}
\definecolor{codegreen}{rgb}{0,0.6,0}
\definecolor{codegray}{rgb}{0.65,0.65,0.65}
\definecolor{codepurple}{rgb}{0.58,0,0.82}
\definecolor{backcolour}{rgb}{0.93, 0.95, 0.96}
\definecolor{keywordcolor}{rgb}{0.23, 0.37, 0.8}
\lstdefinestyle{mystyle}{
    backgroundcolor=\color{backcolour},
    commentstyle=\color{codegreen},
    keywordstyle=\color{keywordcolor}\bf,
    numberstyle=\tiny\color{codegray},
    stringstyle=\color{codepurple},
    basicstyle=\ttfamily\footnotesize,
    breakatwhitespace=false,
    breaklines=true,
    captionpos=b,
    keepspaces=true,
    numbers=left,
    numbersep=5pt,
    showspaces=false,
    showstringspaces=false,
    showtabs=false,
    tabsize=2
}
\lstset{style=mystyle}

\begin{document}
\include{titlepage}
\pagenumbering{arabic}
\setcounter{page}{2}
\tableofcontents
\listoffigures
\clearpage
\section{Создать случайную СЛАУ заданной размерности с заданным числом обусловленности}
\subsection{Ручной расчёт}
\begin{enumerate}
\item Зададим диагональную матрицу, наибольший элемент которой -- желаемое число обусловленности:
\begin{equation}
    \mathbf{D} =
    \begin{pmatrix}
        \mathbf{7} & 0 & 0 & 0 \\
        0 & 2 & 0 & 0 \\
        0 & 0 & 1 & 0 \\
        0 & 0 & 0 & -2
    \end{pmatrix}
\end{equation}

\item Зададим случайный вектор-столбец длины $\mathbf{4}$ и нормируем его, округляя до 3-го знака после запятой:
\begin{equation}
    \mathbf{w}_0 =
    \begin{pmatrix} 8\\ 1\\ 4\\ 9 \end{pmatrix} \implies
    \mathbf{w} = \cfrac{\mathbf{w}_0}{||\mathbf{w}_0||_2} =
    \begin{pmatrix} 0.629\\0.079\\0.314\\0.707\end{pmatrix}
\end{equation}

\item Согласно формуле преобразования Хаусхолдера
\begin{equation}
    \mathbf{Q = E - 2ww}^T
\end{equation}
получим унитарную матрицу $\mathbf{Q}$:
\begin{equation}
    \mathbf{Q} =
    \begin{pmatrix}
        1 & 0 & 0 & 0 \\
        0 & 1 & 0 & 0 \\
        0 & 0 & 1 & 0 \\
        0 & 0 & 0 & 1
    \end{pmatrix} - 2
   \begin{pmatrix} 0.629\\0.079\\0.314\\0.707\end{pmatrix} \cdot
   \begin{pmatrix} 0.629 & 0.079 & 0.314 & 0.707\end{pmatrix} =
    \begin{pmatrix}
         0.209&  -0.099&  -0.395&  -0.889\\
        -0.099&   0.988&  -0.050&  -0.112\\
        -0.395&  -0.050&   0.803&  -0.444\\
        -0.889&  -0.112&  -0.444&   0.000
    \end{pmatrix}
\end{equation}

\item Построим матрицу СЛАУ:
\begin{equation}
    \mathbf{A}=\mathbf{Q}^T\mathbf{DQ} =
    \begin{pmatrix}
        -1.099& -0.520& -1.675& -1.103\\
        -0.520&  1.998&  0.035&  0.417\\
        -1.675&  0.035&  1.348&  2.113\\
        -1.103&  0.417&  2.113&  5.754
    \end{pmatrix} \implies cond(\mathbf{A}) = \mathbf{6.994}
\end{equation}

\item Зададим точное решение и вычислим правую часть уравнения:
\begin{equation}
    \mathbf{x} =
    \begin{pmatrix}
        1\\ 2\\ 3\\ 4
    \end{pmatrix}\implies
    \mathbf{Ax} = \mathbf{b} =
    \begin{pmatrix}-11.576\\5.249\\10.891\\29.086\end{pmatrix}
\end{equation}
\end{enumerate}

\clearpage
\section{Решить СЛАУ методом LU-разложения с выбором главного элемента}
LU-разложение с выбором главного (опорного) элемента это т.н. LUP-разложение, где исходная невырожденная (квадратная с ненулевым определителем) матрица $\mathbf{A}$ представляется в виде:
\begin{equation}
    \mathbf{PA} = \mathbf{LU}
\end{equation}
Здесь матрица $\mathbf{L}$ -- нижнетреугольная с единицами на главной диагонали, $\mathbf{U}$ -- верхнетреугольная общего вида, а $\mathbf{P}$ -- «матрица перестановок», получаемая из единичной матрицы путём перестановки строк или столбцов. $\mathbf{P}$ соответствует вектору перестановок $\mathbf{p}$, в соответствии с которым нужно поменять местами строки $\mathbf{LU}$, чтобы получить $\mathbf{A}$.

\subsection{Алгоритм LUP-разложения}
\begin{enumerate}
    \item $\mathbf{M_0:=A}$, $k$ -- ранг матрицы;
    \item Инициализировать вектор $\mathbf{p}_0$ натуральным рядом $1,2,..., k$;
    \item По каждому столбцу $i := 1,...,k$:
    \begin{enumerate}
        \item В $\mathbf{M_i}$ найти в $i$-м столбце элемент с наибольшим модулем (<<опорный>>). Пусть он находится в $j$-й строке;

        \item В $\mathbf{M_i}$ переставить строки $i \leftrightarrow j$ : $\mathbf{M_{i1}}$;

        \item Переставить в $\mathbf{p}_i$ местами элементы $i \leftrightarrow j$;

        \item В $i$-м столбце элементы лежащие ниже главной диагонали разделить на <<опорный>> элемент: $\mathbf{M_{i2}}$;

        \item Вычесть из каждого элемента, лежащего ниже и правее опорного на $l$-й строке, $m$-м столбце произведение элементов $l$-й строки, $i$-го столбца и $i$-й строки $m$-го столбца: $\mathbf{M_{i3}}$
    \end{enumerate}

    \item $\mathbf{U} := \mathbf{M_{k3}}[i=1..k; j=1..i]$: верхний треугольник, включая диагональ;\\
    $\mathbf{L} := \mathbf{M_{k3}}[i=1..k; j=i+1..k] + \mathbf{E}$: нижний треугольник, на диагонали единицы;\\
    $\mathbf{p} := \mathbf{p}_k ~~~~~\implies ~~~ \blacksquare$
\end{enumerate}

\subsubsection{Комментарий}
В приведённом алгоритме не отражены соображения практической эффективности конкретной реализации алгоритма. Избежать перестановок вектора $\mathbf{p}$ на шаге $3.3$ можно, заполняя неинициализированную ячейку массива значением $j$. Также алгоритм неприменим к матрицам, содержащим столбцы с повторяющимися элементами с наибольшим модулем (в частности, нулевыми), т.к. шаг $3.1$ предполагает, что опорный элемент для столбца единственный.

\subsection{Ручной расчёт}
\begin{enumerate}
    \item Инициализируем вектор $\mathbf{p}_0$ числами натурального ряда:
    \[\mathbf{p}_0 = \begin{pmatrix} 1 \\ 2 \\ 3 \\ 4 \end{pmatrix}\]

    \item В первом столбце $\mathbf{M}_0 = \mathbf{A}$ найдём элемент с наибольшим модулем:
    \[\mathbf{M}_0 =
    \begin{pmatrix}
        -1.099& -0.520& -1.675& -1.103\\
        -0.520&  1.998&  0.035&  0.417\\
       \mathbf{-1.675}&  0.035&  1.348&  2.113\\
        -1.103&  0.417&  2.113&  5.754
    \end{pmatrix}\]

    \item Переставим строку с опорным элементом так, чтобы он оказался на главной диагонали, отразив соответствующую перестановку в векторе перестановок:
    \[\mathbf{M}_{11} =
    \begin{pmatrix}
        \mathbf{-1.675}& \mathbf{ 0.035}& \mathbf{ 1.348}& \mathbf{ 2.113}\\
        -0.520&  1.998&  0.035&  0.417\\
        \mathit{-1.099}& \mathit{-0.520}& \mathit{-1.675}& \mathit{-1.103}\\
        -1.103&  0.417&  2.113&  5.754
    \end{pmatrix}; \hspace{1cm}
    \mathbf{p}_1 = \begin{pmatrix} \mathbf{3} \\ 2 \\ \mathit{1} \\ 4 \end{pmatrix}\]

    \item Все элементы первого столбца лежащие под опорным элементом (ниже 1-й строки) разделим на него:
    \[\mathbf{M}_{12} =
    \begin{pmatrix}
        \mathbf{-1.675}& 0.035& 1.348& 2.113\\
        \frac{-0.520}{-1.675}&  1.998&  0.035&  0.417\\
        \frac{-1.099}{-1.675}& -0.520& -1.675& -1.103\\
        \frac{-1.103}{-1.675}&  0.417&  2.113&  5.754
    \end{pmatrix} =
    \begin{pmatrix}
        \mathbf{-1.675}& 0.035& 1.348& 2.113\\
        \mathit{0.310}&  1.998&  0.035&  0.417\\
        \mathit{0.656}& -0.520& -1.675& -1.103\\
        \mathit{0.659}&  0.417&  2.113&  5.754
    \end{pmatrix}\]

    \item Далее из элементов каждого следующего столбца, лежащих ниже опорной (1-й) строки вычтем произведение соответствующего элемента первого столбца и первой строки:
    \[\mathbf{M}_{13} = \begin{pmatrix}
               -1.675 &  \mathbf{0.035}&  \mathbf{1.348}& \mathbf{2.113}\\
        \mathbf{0.310}&  1.998-0.310\cdot0.035&  0.035-0.310\cdot1.348&  0.417-0.310\cdot2.113\\
        \mathbf{0.656}& -0.520-0.656\cdot0.035& -1.675-0.656\cdot1.348& -1.103-0.656\cdot2.113\\
        \mathbf{0.659}&  0.417-0.659\cdot0.035&  2.113-0.659\cdot1.348&  5.754-0.659\cdot2.113
    \end{pmatrix}=\]
    \[=\begin{pmatrix}
      -1.675&  0.035&  1.348&  2.113\\
       0.310&  1.987& -0.383& -0.238\\
       0.656& -0.543& -2.559& -2.489\\
       0.659&  0.394&  1.225&  4.362
    \end{pmatrix}\]

    \item Далее переходим ко второму столбцу и второй строке, повторяя пп. 2-5. В нашем случае, элемент с наибольшим модулем уже находится на главной диагонали, так что переходим к шагу деления элементов матрицы под элементом на главной диагонали:
    \[\mathbf{M}_{22} =
    \begin{pmatrix}
       -1.675&  0.035&  1.348&  2.113\\
        0.310&  \mathbf{1.987}& -0.383& -0.238\\
        0.656& \frac{-0.543}{1.987}& -2.559& -2.489\\
        0.659&  \frac{0.394}{1.987}&  1.225&  4.362
    \end{pmatrix}=
    \begin{pmatrix}
      -1.675&          0.035 &  1.348&  2.113\\
       0.310& \mathbf{ 1.987}& -0.383& -0.238\\
       0.656& \mathit{-0.273}& -2.559& -2.489\\
       0.659& \mathit{ 0.198}&  1.225&  4.362
    \end{pmatrix}\]

    \item Шаг вычитания:
    \[\mathbf{M}_{23} =
     \begin{pmatrix}
      -1.675&          0.035 &  1.348&  2.113\\
       0.310& \mathbf{ 1.987}& \mathbf{-0.383}& \mathbf{-0.238}\\
       0.656& \mathbf{-0.273}& -2.559-0.273\cdot0.383& -2.489-0.273\cdot0.238\\
       0.659& \mathbf{ 0.198}&  1.225+0.198\cdot0.383&  4.362+0.198\cdot0.238
    \end{pmatrix} =
    \begin{pmatrix}
       -1.675&  0.035&   1.348&   2.113\\
        0.310&  1.987&  -0.383&  -0.239\\
        0.656& -0.273&  -2.664&  -2.555\\
        0.659&  0.198&   1.301&   4.410
    \end{pmatrix}\]

    \item В третьем столбце элемент с наибольшим модулем так же уже находится на диагонали. Переходим к шагу деления:
     \[\mathbf{M}_{32} =
     \begin{pmatrix}
      -1.675&  0.035&   1.348&   2.113\\
       0.310&  1.987&  -0.383&  -0.239\\
       0.656& -0.273&  \mathbf{-2.664}&  -2.555\\
       0.659&  0.198&  \frac{1.301}{-2.664}&   4.410
     \end{pmatrix} =
     \begin{pmatrix}
      -1.675&  0.035&   1.348&   2.113\\
       0.310&  1.987&  -0.383&  -0.239\\
       0.656& -0.273&  -2.664&  -2.555\\
       0.659&  0.198&  \mathit{-0.488}&   4.410
     \end{pmatrix}\]

     \item Шаг вычитания:
     \[\mathbf{M}_{33} =
     \begin{pmatrix}
       -1.675&  0.035&   1.348&   2.113\\
        0.310&  1.987&  -0.383&  -0.239\\
        0.656& -0.273&  -2.664&  \mathbf{-2.555}\\
        0.659&  0.198&  \mathbf{-0.488}&   4.410-2.555\cdot0.488
     \end{pmatrix} =
     \begin{pmatrix}
       -1.675&  0.035&   1.348&   2.113\\
        0.310&  1.987&  -0.383&  -0.239\\
        0.656& -0.273&  -2.664&  -2.555\\
        0.659&  0.198&  -0.488&   3.162
     \end{pmatrix}\]

     \item Шаг перестановок по последнему столбцу пропускаем, т.к. элемент с наибольшим модулем уже на главной диагонали.
\end{enumerate}
Полученная матрица $\mathbf{M}_{33}=\mathbf{M}$ вкупе с вектором перестановок $\mathbf{p}_1=\mathbf{p}$ являются компактным представлением LUP-разложения матрицы $\mathbf{A}$. По-отдельности матрицы $\mathbf{L}$ и $\mathbf{U}$ можно получить пользуясь соотношением:
\begin{align}
    \mathbf{M} = \mathbf{L} - \mathbf{E} + \mathbf{U}
\end{align}
То есть:
\begin{equation}
    \mathbf{L} =
    \begin{pmatrix}
        1    &  0    &   0    &   0\\
        0.310&  1    &   0    &   0\\
        0.656& -0.273&   1    &   0\\
        0.659&  0.198&  -0.488&   1
    \end{pmatrix};~~~\mathbf{U} =
    \begin{pmatrix}
       -1.675&  0.035&   1.348&   2.113\\
        0    &  1.987&  -0.383&  -0.239\\
        0    &  0    &  -2.664&  -2.555\\
        0    &  0    &   0    &   3.162
    \end{pmatrix};~~~
    \mathbf{p} = \begin{pmatrix} 3 \\ 2 \\ 1 \\ 4  \end{pmatrix}
\end{equation}

\subsection{Решение СЛАУ с помощью полученного разложения}
\begin{enumerate}
    \item Прямая подстановка -- решение уравнения $\mathbf{Ly} = \mathbf{Pb}=\mathbf{b}_p$ относительно $\mathbf{y}$:
    \[ (\mathbf{L}|\mathbf{b}_p) =
    \left(
    \begin{array}{cccc|c}
        1    &  0    &   0    &   0& 10.891 \\
        0.310&  1    &   0    &   0&  5.249\\
        0.656& -0.273&   1    &   0&-11.576\\
        0.659&  0.198&  -0.488&   1& 29.086
    \end{array}\right) \implies
    \mathbf{y} = \begin{pmatrix} 10.891 \\1.873 \\ -18.209 \\ 12.652 \end{pmatrix}; \]
    \item Обратная подстановка -- решение уравнения $\mathbf{U}\widetilde{\mathbf{x}} = \mathbf{y}$:
    \[ (\mathbf{U}|\mathbf{y}) =
    \left( \begin{array}{cccc|c}
      -1.675&  0.035&   1.348&   2.113& 10.891\\
       0    &  1.987&  -0.383&  -0.239&  1.873\\
       0    &  0    &  -2.664&  -2.555&-18.209\\
       0    &  0    &   0    &   3.162& 12.652
    \end{array} \right) \implies
    \widetilde{\mathbf{x}} = \begin{pmatrix} 1.000\\2.002\\2.998\\4.001\end{pmatrix}; \]

    \item Рассчитаем вектор невязки по формуле: $\mathbf{r}=\mathbf{b}-\textbf{A}\widetilde{\mathbf{x}}$
    \[ \mathbf{r}=
    \begin{pmatrix}-11.576\\5.249\\10.891\\29.086\end{pmatrix} -
    \begin{pmatrix}
        -1.099& -0.520& -1.675& -1.103\\
        -0.520&  1.998&  0.035&  0.417\\
        -1.675&  0.035&  1.348&  2.113\\
        -1.103&  0.417&  2.113&  5.754
    \end{pmatrix} \cdot
    \begin{pmatrix} 1.000\\2.002\\2.998\\4.001\end{pmatrix} =
    \begin{pmatrix}  -0.001\\-0.004\\0.001\\-0.002 \end{pmatrix}; ~~~~ ||\mathbf{r}||_2 = \mathbf{0.005} \]
\end{enumerate}

\clearpage
\section{Воспользоваться запрограммированным методом и операцией Matlab «$\backslash$» на тестовых матрицах с нулевым определителем}
\subsection{Запрограммированный метод}
\subsubsection{Реализация в C++}
Основная вызываемая в программе функция выглядит так:
\begin{lstlisting}[language=c++]
LU_result LUP(Mtr M){
    size_t N = M.size();
    vInt perm(N); for (int i = 0; i < N; i++) perm[i] = i; // perm = {0,1,2,...,N-1}

    for (size_t i=0; i<N; i++){ // Iterating over each row
        find_max_and_swap(M, perm, i);
        div_under_main_diag(M, i);
        subtract_product(M, i);
    }
    auto [L, U] = split_LUP(M);
    return {L, U, perm};
}
\end{lstlisting}
Она проходит по каждой строке матрицы и последовательно применяет 3 шага, описанных в (2.1): шаг поиска максимального элемента (опорного) в столбце и перестановки двух строк, чтобы опорный оказался на главной диагонали (одновременно с этим происходит перестановка в векторе перестановок), шаг деления элементов столбца под опорным элементом на его значение и шаг вычитания произведений соответствующих элементов из подматрицы ниже и правее опорного элемента. После этих итераций необходимо разделить полученную матрицу на $\mathbf{L}$ и $\mathbf{U}$ части и вернуть их вместе с вектором перестановок $\mathbf{p}$.
\subsubsection{Модульная структура программы}
В программной реализации абстрагированы в подфункции многие часто используемые действия в линейной алгебре:
\begin{lstlisting}[language=c++]
using namespace std;
using Vec = vector<double>;
using vInt = vector<int>;
using Mtr = vector<Vec>;
using LU_result = tuple <Mtr, Mtr, vInt>;
// Matrix algebra:
Vec sum(const Vec& V1, const Vec& V2, double c1=1, double c2=1); // c1*V1 + c2*V2
Mtr sum(const Mtr& M1, const Mtr& M2, double c1=1, double c2=1); // c1*M1 + c2*M2
Vec mul(double scalar, const Vec& V);
Mtr mul(double scalar, const Mtr& M);
Vec mul(const Mtr& M,  const Vec& x);
Mtr mul(const Mtr& M1, const Mtr& M2);
// LU:
LU_result LUP(Mtr M);
void find_max_and_swap(Mtr& M, vInt& perm, int i);
void div_under_main_diag(Mtr& M, int i);
void subtract_product(Mtr& M, int i);
pair<Mtr, Mtr> split_LUP(Mtr& U);
// SLAE:
Vec solveLinearEquation(const Mtr& L, const Mtr& U, const vInt& permutation, const Vec& b);
Vec residual(const Mtr& A, const Vec& x, const Vec& b);
// Special:
Mtr inv(const Mtr& M);
bool issquare(const Mtr& M);
double det(const Mtr& M);
\end{lstlisting}
\subsubsection{Проверка матриц A и B:}
{\singlespacing
\begin{verbatim}
Проверка матрицы A:
A:                        |x:         |b:
1           2           3 |         x1|    6.40157
4           5           6 |         x2|    3.47909
7           8           9 |         x3|   -1.09412

Определитель:
0

Для решения системы нужно произвести LU-разложение матрицы:
LU-разложение:

L:                        |U:
1           0          0  |7        8          9
0.142857    1          0  |0        0.857143   1.71429
0.571429    0.5        1  |0        0          1.11022e-16

Вектор перестановок:
[2, 0, 1]

Соберём матрицу назад:
1           2           3
4           5           6
7           8           9

Вычислим вектор неизвестных, основываясь на LU-разложении:
[7.43422e+15, -1.48684e+16, 7.43422e+15]

Вектор невязки найденного решения:
[-2.40157, -11.4791, 1.09412]

2-норма вектора невязки:
11.7785

Проверка матрицы B:
B:                            |x:         |b:
1e+08       2e+08       3e+08 |         x1|    6.40157
4e+08       5e+08       6e+08 |         x2|    3.47909
7e+08       8e+08       9e+08 |         x3|   -1.09412

Определитель:
-1.07374e+09

Для решения системы нужно произвести LU-разложение матрицы:
i = 2, pivot = 0
Ошибка: Невозможно найти максимальнный элемент в нулевом столбце!
\end{verbatim}}
\subsubsection{Комментарий:}
\begin{enumerate}
    \item В случае с матрицей $\mathbf{A}$ программа точно вычислила определитель (по наивной рекурсивной формуле) и успешно осуществила LU-разложение. Однако при дальнейшем решении СЛАУ невязка полученного решения оказалась очень большой по причине заведомо плохой обусловленности матрицы  $\mathbf{A}$.

    \item В случае с матрицей $\mathbf{B}$ программа не получила точного значения определителя вследствие ошибки округления чисел с плавающей точкой (аналогичные результаты можно получить в математических пакетах, если попытаться вычислить определитель матрицы $\mathtt{det((1e8+0.1) * A)}$). При попытке же прозвести LU-разложение программа завершилась с ошибкой, т.к. в третьем столбце не был найден максимальный элемент (все элементы равны нулю), а значит невозможно осуществить шаг перестановки и следующий за ним шаг деления.
\end{enumerate}

\subsection{Оператор Matlab <<$\backslash$>>:}
\subsubsection{Код программы в Matlab:}
\begin{lstlisting}[language=octave]
A = [
1 2 3
4 5 6
7 8 9];
B = 1e8*A;
b = rand(n, 1);
% Warning: Matrix is singular to working precision.
% Warning: Matrix is close to singular or badly scaled.
xA = A \ b; xB = B \ b;
disp(A);
disp(['Det(A): ' num2str(det(A))]);
disp(['Cond(A): ' num2str(cond(A))]);
disp(['||rA||: ' num2str(norm(b-A*xA))]);

disp(B);
disp(['Det(B): ' num2str(det(B))]);
disp(['Cond(B): ' num2str(cond(B))]);
disp(['||rB||: ' num2str(norm(b-B*xB))]);
\end{lstlisting}
\subsubsection{Результат:}
{\singlespacing
\begin{verbatim}
Warning: Matrix is close to singular or badly scaled.
> In LAB_0_4__3 (line 19)
1     2     3
4     5     6
7     8     9
Det(A): 6.6613e-16
Cond(A): 1.143944118188076e+17
||rA||: 4.3504
Warning: Matrix is singular to working precision.
> In LAB_0_4__3 (line 26)
100000000   200000000   300000000
400000000   500000000   600000000
700000000   800000000   900000000
Det(B): 0
Cond(B): 2.556914355281162e+16
||rB||: NaN
\end{verbatim}}
\subsubsection{Комментарий:}
Полученные теперь результаты, отличаясь в деталях от прежних, в главном сходятся:
\begin{enumerate}
    \item Определитель матрицы $\mathbf{A}$ получился не нулевым, похоже внутренняя реализация этой функции в Matlab опирается на перемножение элементов диагонали матрицы $\mathbf{U}$ после LU-разложения. Если посмотреть на них и в реализации на C++, то их произведение так же получится отличным от нуля (пусть и близким к машинному эпсилон). Matlab получил некоторый результат, невязка которого так же оставляет желать лучшего;

    \item Определитель же второй матрицы посчитан в точности равным 0, поэтому внутренняя логика Matlab остановила расчёт корней СЛАУ и численного ответа получено не было.

    \item В обоих случаях решения, предоставленные программами, формально неверны, т.к. определители обеих матриц должны быть равны нулю {\bf в точности}, что означает неопределённость чисел обусловленности и наличие нетривиальных решений СЛАУ (или их полное отсутствие).
\end{enumerate}

\clearpage
\section{В MatLab Решить СЛАУ с матрицами Гильберта}
\subsection{Теория}
Матрицы Гильберта -- особый класс квадратных матриц, использующийся для проверки алгоритмов и оценки численной стабильности. Они представляют собой модель системы линейных уравнений с плохой обусловленностью. Элементы матрицы задаются формулой:
\begin{equation}
    H_{ij}={\frac {1}{i+j-1}},i,j=1,2,3,...,n
\end{equation}
Ожидается, что решение прямыми методами СЛАУ с матрицами Гильберта будет иметь большую погрешность из-за их плохой обусловленности.
\subsection{Код программы в Matlab:}
\begin{lstlisting}[language=octave]
for n = [5, 10, 15]
    A = hilb(n);
    x_exact = randn(n, 1);
    b = A * x_exact;
    x = A \ b;

    disp(['Size: ' num2str(n)]);
    disp(['cond(A) = ' num2str(cond(A))]);
    disp(['||eps|| = ' num2str(norm(x - x_exact))]);
    disp(['||r|| = ' num2str(norm(A*x - b))]);
    disp('-----------------------------');
end

\end{lstlisting}
\subsection{Результат:}
{\singlespacing
\begin{verbatim}
Size: 5
cond(A) = 476607.2502
||eps|| = 2.5412e-11
||r|| = 1.1102e-16
-----------------------------
Size: 10
cond(A) = 16024909625167.58
||eps|| = 0.00022917
||r|| = 2.8339e-16
-----------------------------
Warning: Matrix is close to singular or badly scaled. Results may be inaccurate. RCOND =  1.543404e-18.
> In LAB_0_4__4 (line 7)

Size: 15
cond(A) = 3.378778714747708e+17
||eps|| = 5.8704
||r|| = 3.5517e-16
-----------------------------
\end{verbatim}}
\subsection{Комментарий:}
Стремительный рост числа обусловленности матриц Гильберта как $O\left(\frac{(1+\sqrt{2})^{4n}}{\sqrt{n}}\right)$ и рост фактической ошибки решения не влечёт роста нормы невязки решения для случайного вектора правой части.

\clearpage
\section{Построить график временных затрат решения СЛАУ для операции Matlab «$\backslash$» в зависимости от размера матрицы}
\subsection{Код программы в Matlab:}
\begin{lstlisting}[language=octave]
repeats = 10;  n_max = 35;
matrix_sizes = round(logspace(log10(10), log10(2000), n_max));
times = zeros(1, n_max);

for repeat = 1:repeats
    for i = 1:n_max
        n = matrix_sizes(i);  A = fix_cond(n, 10);  b = randn(n, 1);
        tic;  x = A \ b;
        times(i) = times(i) + toc;
    end
end
times = times / repeats;

loglog(matrix_sizes, times, 'o');
xlabel('Matrix size'); ylabel('Eval time, s.');
title("Benchmarking SLAE solving using '\\' operator");
legend("Averaged by " + string(repeats))
grid on; hold on;
\end{lstlisting}
\subsection{Графики}
\begin{figure}[H]
    \centering
    \caption{График зависимости времени решения СЛАУ от размера случайной матрицы. cond = 10.}
    \includegraphics[width=1\linewidth]{../pics/task_5}
    \label{pic:1}
\end{figure}
\begin{figure}[H]
    \centering
    \caption{Подгонка кривой степенной зависимости в интерактивном режиме.}
    \includegraphics[width=1\linewidth]{../pics/cfit}
    \label{pic:2}
\end{figure}
\subsection{Комментарий:}
В аппроксимирующей функции $t=a\cdot \text{matrix\_size}^b$ рассчитанный показатель степени $b=2.338$.

\end{document}