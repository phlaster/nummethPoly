\documentclass[a4paper]{article}
\usepackage[russian]{babel}
\usepackage[utf8]{inputenc}
\usepackage{setspace}
\setstretch{1.5}
\usepackage{anyfontsize}
\usepackage{amsmath, amsfonts, amssymb, amsthm, mathtools}
\usepackage{geometry}
\usepackage{graphicx}
\usepackage{wrapfig}
\usepackage{xcolor}
\usepackage{multicol}
\usepackage{listings}
\geometry{a4paper, portrait, margin=10mm, bmargin=15mm, tmargin=15mm}
\definecolor{codegreen}{rgb}{0,0.6,0}
\definecolor{codegray}{rgb}{0.65,0.65,0.65}
\definecolor{codepurple}{rgb}{0.58,0,0.82}
\definecolor{backcolour}{rgb}{0.93, 0.95, 0.96}
\definecolor{keywordcolor}{rgb}{0.23, 0.37, 0.8}
\lstdefinestyle{mystyle}{
    backgroundcolor=\color{backcolour},
    commentstyle=\color{codegreen},
    keywordstyle=\color{keywordcolor}\bf,
    numberstyle=\tiny\color{codegray},
    stringstyle=\color{codepurple},
    basicstyle=\ttfamily\footnotesize,
    breakatwhitespace=false,
    breaklines=true,
    captionpos=b,
    keepspaces=true,
    numbers=left,
    numbersep=5pt,
    showspaces=false,
    showstringspaces=false,
    showtabs=false,
    tabsize=2
}
\lstset{style=mystyle}


\begin{document}
\include{titlepage}
\newpage
\pagenumbering{arabic}
\setcounter{page}{2}

{\Large{\textbf{A. Приближение данных полиномиальным МНК}}}\\
{\textbf{1. Приближение данных полиномиальным МНК при помощи функции \texttt{polyfit}}}
\begin{lstlisting}[language=Octave]
a=0.5; b=4.5; x=a:0.05:b;
p = [1, -8.1, 19.04, -10.56];
p_legend = 'y = x^3 - 8.1x^2 + 19.04x - 10.56';
y = polyval(p, x);
plot(x, y, 'b'); grid on;
legend(p_legend); title('Polynomial');
saveas(gcf, '1.png'); hold off;
\end{lstlisting}
\begin{center}
    \includegraphics[width=1\linewidth]{1}
\end{center}

\newpage
\begin{lstlisting}[language=Octave, firstnumber=8]
x = linspace(a,b,150); t = linspace(a,b,1001);
y = polyval(p,x) + randn(size(x));
p1 = polyfit(x,y,1); y1 = polyval(p1,t);
err1 = y - polyval(p1,x);

subplot(2,1,1);
plot(x, y, 'b.');
hold on; grid on;
plot(t, y1, 'r');
xlabel('x'); ylabel('y');
title('Approximation with polynomial with degree 1'); hold off;

subplot(2, 1, 2);
plot(x, err1,'r.'); grid on;
xlabel('x'); ylabel('Err'); title('max error ', max(abs(err1)));
saveas(gcf, '2.png');
\end{lstlisting}
\begin{center}
    \includegraphics[width=1\linewidth]{2}
\end{center}
{\singlespacing
\begin{verbatim}
p1 =

   -0.3013    1.9069
\end{verbatim}}


\newpage
\begin{lstlisting}[language=Octave, firstnumber=24]
p2 = polyfit(x,y,2); y2 = polyval(p2,t);
err2 = y - polyval(p2, x);

subplot(2,1,1);
plot(x, y, 'b.');
hold on; grid on;
plot(t, y2, 'r');
xlabel('x'); ylabel('y');
title('Approximation with polynomial with degree 2'); hold off;

subplot(2, 1, 2);
plot(x, err2,'r.'); grid on;
xlabel('x'); ylabel('Err'); title('max error ', max(abs(err2)));
saveas(gcf, '3.png');
\end{lstlisting}
\begin{center}
    \includegraphics[width=1\linewidth]{3}
\end{center}
{\singlespacing
\begin{verbatim}
p2 =

   -0.6739    3.0683   -1.3945
\end{verbatim}}


\newpage
\begin{lstlisting}[language=Octave, firstnumber=38]
p3 = polyfit(x,y,3); y3 = polyval(p3,t);
err3 = y - polyval(p3, x);

subplot(2,1,1);
plot(x, y, 'b.');
hold on; grid on;
plot(t, y3, 'r');
xlabel('x'); ylabel('y');
title('Approximation with polynomial with degree 3'); hold off;

subplot(2, 1, 2);
plot(x, err3,'r.'); grid on;
xlabel('x'); ylabel('Err'); title('max error ', max(abs(err3)));
saveas(gcf, '4.png');
\end{lstlisting}
\begin{center}
    \includegraphics[width=1\linewidth]{4}
\end{center}
{\singlespacing
\begin{verbatim}
p3 =

    0.9653   -7.9136   18.8197  -10.6079
\end{verbatim}}



\newpage
\begin{lstlisting}[language=Octave, firstnumber=52]
p4 = polyfit(x,y,4); y4 = polyval(p4,t);
err4 = y - polyval(p4, x);

subplot(2,1,1);
plot(x, y, 'b.');
hold on; grid on;
plot(t, y4, 'r');
xlabel('x'); ylabel('y');
title('Approximation with polynomial with degree 4'); hold off;

subplot(2, 1, 2);
plot(x, err4,'r.'); grid on;
xlabel('x'); ylabel('Err'); title('max error ', max(abs(err4)));
saveas(gcf, '5.png');
\end{lstlisting}
\begin{center}
    \includegraphics[width=1\linewidth]{5}
\end{center}
{\singlespacing
\begin{verbatim}
p4 =

   -0.0474    1.4392   -9.5262   20.9586  -11.4969
\end{verbatim}}



\newpage
{\textbf{2. Приближение данных полиномиальным МНК в приложении \texttt{cftool}}}\\
Первая степерь:
\begin{center}
    \includegraphics[width=1\linewidth]{6}
\end{center}
Вторая степерь:
\begin{center}
    \includegraphics[width=1\linewidth]{7}
\end{center}

\newpage

Третья степерь:
\begin{center}
    \includegraphics[width=1\linewidth]{8}
\end{center}
Четвёртая степерь:
\begin{center}
    \includegraphics[width=1\linewidth]{9}
\end{center}
Коэффициенты $p_i$, полученные в приложении \texttt{cftool} соответствуют -- полученным с помощью функции \texttt{polyfit}.
\newpage

{\Large{\textbf{Б. Приближение данных с выбросами}}}\\
{\textbf{3. Приближение данных с выбросами в приложении \texttt{cftool}}}
\begin{lstlisting}[language=Octave, firstnumber=66]
y1 = y; y1(1:10:end) = 10;
\end{lstlisting}
Robust: Off
\begin{center}
    \includegraphics[width=1\linewidth]{10}
\end{center}
Robust: Bisquare
\begin{center}
    \includegraphics[width=1\linewidth]{11}
\end{center}

\newpage
Robust: LAR
\begin{center}
    \includegraphics[width=1\linewidth]{12}
\end{center}

\newpage

{\Large{\textbf{В. Приближение данных сглаживающими сплайнами}}}\\
{\textbf{4. Приближение данных сглаживающими сплайнами в \texttt{cftool}}}\\
Smoothing Parameter: 0
\begin{center}
    \includegraphics[width=1\linewidth]{13}
\end{center}
Smoothing Parameter: 0.5
\begin{center}
    \includegraphics[width=1\linewidth]{14}
\end{center}

\newpage
Smoothing Parameter: 0.9
\begin{center}
    \includegraphics[width=1\linewidth]{15}
\end{center}
Smoothing Parameter: 0.999
\begin{center}
    \includegraphics[width=1\linewidth]{16}
\end{center}

\newpage
Smoothing Parameter: 0.9999
\begin{center}
    \includegraphics[width=1\linewidth]{17}
\end{center}
Smoothing Parameter: 1
\begin{center}
    \includegraphics[width=1\linewidth]{18}
\end{center}


\newpage
{\textbf{5. Приближение данные сглаживающим сплайном при помощи функции \texttt{csaps}}}
\begin{lstlisting}[language=Octave, firstnumber=67]
plot(x,y,'b.'); hold on; grid on;
pp = csaps(x, y, 0.8);
yt = fnval(pp, t); plot(t, yt,'r');
saveas(gcf, '19.png');
\end{lstlisting}
\begin{center}
    \includegraphics[width=1\linewidth]{19}
\end{center}

\newpage
{\Large{\textbf{Г. Приближение данных нелинейными моделями МНК}}}\\
{\textbf{6. Приближение данных нелинейными моделями МНК в приложении \texttt{cftool}}}
\begin{lstlisting}[language=Octave, firstnumber=72]
a0=-3;  b0=3;  x=a0:0.05:b0;

a=2.0;  b=0.8;  c=2.8;  d=0.3;
y = exp(a-b*x.^2) .* cos(sqrt(d+c * x.^2));
plot(x,y,'b'); grid on;
saveas(gcf, '20.png');
\end{lstlisting}
\begin{center}
    \includegraphics[width=0.5\linewidth]{20}
\end{center}
\begin{lstlisting}[language=Octave, firstnumber=78]
x=linspace(a0, b0, 150); rnd(1);
y=exp(a-b*x.^2) .* cos(sqrt(d+c * x.^2)) + randn(size(x));
\end{lstlisting}
Degree: 6
\begin{center}
    \includegraphics[width=1\linewidth]{21}
\end{center}

\newpage
Оптимальная функция: $y=\exp(a-bx^2) \cdot \cos(\sqrt{d+c x^2})$, где: $a=2;b=0.8;c=2.8;d=0.3;$

Приближение: $f(x) = \exp(-bx^2) \cdot \cos(\sqrt{vx^2})$, стартовые параметры: b\textleftarrow 1; c\textleftarrow 3
\begin{center}
    \includegraphics[width=1\linewidth]{22}
\end{center}
Приближение: $f(x) =\exp(a-bx^2) \cdot \cos(\sqrt{d+c x^2})$, стартовые параметры: a\textleftarrow 2; b\textleftarrow 1; c\textleftarrow 3; d\textleftarrow 0;
\begin{center}
    \includegraphics[width=1\linewidth]{23}
\end{center}

\newpage
{\textbf{7. Приближение данных нелинейными моделями МНК при помощи функции fit}}
\begin{lstlisting}[language=Octave, firstnumber=80]
NlModel = 'exp(a-b*x*x).*cos(sqrt(d+c*x*x))';
StartPoint=[2,1,3,0];
f1 = fit(x',y',NlModel,'Start', StartPoint);
plot(f1, x, y); grid on;
saveas(gcf, '24.png');
\end{lstlisting}
\begin{center}
    \includegraphics[width=1\linewidth]{24}
\end{center}
{\singlespacing
\begin{verbatim}
f1 =

    General model:
    f1(x) = exp(a-b*x*x).*cos(sqrt(d+c*x*x))
    Coefficients (with 95% confidence bounds):
      a =       4.297  (-383.4, 392)
      b =       1.165  (-13.46, 15.79)
      c =      0.3782  (-142.4, 143.1)
      d =       2.196  (-100.1, 104.5)
\end{verbatim}}
Коэффициенты $a,b,c,d$, полученные в приложении \texttt{cftool} соответствуют -- полученным с помощью функции \texttt{fit}.

\end{document}