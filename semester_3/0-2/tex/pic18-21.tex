\begin{figure}[H]
    \centering
    \caption{$~~\varepsilon_U\le0.1885;~~~\varepsilon_T\le0.1793$}
    \includegraphics[width=0.85\linewidth]{pics/pic18}
    \label{pic:18}
\end{figure}
\begin{figure}[H]
    \centering
    \caption{~~$\varepsilon_U\le0.1304;~~~\varepsilon_T\le0.0958$}
    \includegraphics[width=0.85\linewidth]{pics/pic19}
    \label{pic:19}
\end{figure}
\begin{figure}[H]
    \centering
    \caption{~~$\varepsilon_U\le1.174\cdot10^{-3};~~~\varepsilon_T\le1.384\cdot10^{-4}$, минимальная ошибка убывает не так быстро, как у полинома Лагранжа, но профиль ошибки более равномерный:}
    \includegraphics[width=0.85\linewidth]{pics/pic20}
    \label{pic:20}
\end{figure}
\begin{figure}[H]
    \centering
    \caption{~~$\varepsilon_U\le6.154\cdot10^{-8};~~~\varepsilon_T\le2.334\cdot10^{-12}$}
    \includegraphics[width=0.85\linewidth]{pics/pic21}
    \label{pic:21}
\end{figure}