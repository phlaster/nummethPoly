\begin{figure}[H]
    \centering
    \caption{$\varepsilon \le 15.8236$}
    \includegraphics[width=0.85\linewidth]{pics/pic13}
    \label{pic:13}
\end{figure}
\begin{figure}[H]
    \centering
    \caption{В отличие от метода Лагранжа, кубический сплайн Эрмита не совпадает с полиномом высшей степени, даже когда количество $n>5$, $\varepsilon \le 4.6176$}
    \includegraphics[width=0.9\linewidth]{pics/pic14}
    \label{pic:14}
\end{figure}
\begin{figure}[H]
    \centering
    \caption{$\varepsilon \le 0.5379$}
    \includegraphics[width=0.9\linewidth]{pics/pic15}
    \label{pic:15}
\end{figure}
\begin{figure}[H]
    \centering
    \caption{Сводный график зависимости максимальной ошибки на интервале от количества узлов на нём.}
    \includegraphics[width=1\linewidth]{pics/pic16}
    \label{pic:16}
\end{figure}