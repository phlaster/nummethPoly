\begin{figure}[H]
    \centering
    \caption{$\varepsilon \le 12.5019$}
    \includegraphics[width=0.8\linewidth]{pics/pic5}
    \label{pic:5}
\end{figure}

\begin{figure}[H]
    \centering
    \caption{При интерполяции на 6 узлах полином Лагранжа {\bf в точности} совпадает с многочленом 5-й степени, добавлять больше узлов не имеет смысла. $\varepsilon~\le~1.42109\cdot10^{-14}$}
    \includegraphics[width=0.8\linewidth]{pics/pic6}
    \label{pic:6}
\end{figure}

\begin{figure}[H]
    \centering
    \caption{С ростом числа узлов интерполяции так же отметим проявление феномена Рунге. $\varepsilon~\le~4.57866\cdot10^{-9}$}
    \includegraphics[width=0.8\linewidth]{pics/pic7}
    \label{pic:7}
\end{figure}

\begin{figure}[H]
    \centering
    \caption{Для обеих функций зависимость максимальной ошибки интерполяции от количества узлов полинома Лагранжа. В качестве случайной точки выбрано значение $t = 0.9718075809237104$}
    \includegraphics[width=0.8\linewidth]{pics/pic8}
    \label{pic:8}
\end{figure}