\begin{figure}[H]
    \centering
    \caption{График интерполяции $f_1(x)$ сплайном Эрмита по 4 узлам, $\varepsilon \le 0.2986$}
    \includegraphics[width=1\linewidth]{pics/pic9}
    \label{pic:9}
\end{figure}
На среднем отрезке сплайн пересекает график функции и ошибка устремляется к $0$.
\begin{figure}[H]
    \centering
    \caption{Та же функция, 9 узлов, ошибка $\varepsilon \le 0.0681$}
    \includegraphics[width=0.9\linewidth]{pics/pic10}
    \label{pic:10}
\end{figure}
Теперь почти на каждом междоузлии есть сплайн пересекает график функции.
\begin{figure}[H]
    \centering
    \caption{Ошибка убывает не так быстро, как у полинома Лагранжа ($\varepsilon \le 0.004229$) Но распределяется более равномерно}
    \includegraphics[width=0.9\linewidth]{pics/pic11}
    \label{pic:11}
\end{figure}
\begin{figure}[H]
    \centering
    \caption{В отличие от глобальной интерполяции Лагранжа, сплайн в меньшей степени подвержен феномену Рунге. $\varepsilon \le 0.001096$}
    \includegraphics[width=0.9\linewidth]{pics/pic12}
    \label{pic:12}
\end{figure}