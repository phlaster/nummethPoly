\documentclass[a4paper, 14pt]{article}
\usepackage[russian]{babel}
\usepackage[utf8]{inputenc}
\usepackage{setspace}
\setstretch{1.3}
\usepackage{anyfontsize}
\usepackage{amsmath, amsfonts, amssymb, amsthm, mathtools}
\usepackage{geometry}
\usepackage{graphicx}
\usepackage{wrapfig}
\geometry{a4paper, portrait, margin=10mm, bmargin=15mm, tmargin=15mm}

%%%%%%%%%%%%%%%%%%%%%%%%%%%%%%%%%%%%%%%%%%%%%%%%%%%%%%%%%%%%%%%%%%%%%%%%%%%%
%%%%%%%%%%%%%%%%%%%%%%%%%%%%%%%%%%%%%%%%%%%%%%%%%%%%%%%%%%%%%%%%%%%%%%%%%%%%
%%%%%%%%%%%%%%%%%%%%%%%%%%%%%%%%%%%%%%%%%%%%%%%%%%%%%%%%%%%%%%%%%%%%%%%%%%%%

\begin{document}

\include{titlepage}

%%%%%%%%%%%%%%%%%%%%%%%%%%%%%%%%%%%%%%%%%%%%%%%%%%%%%%%%%%%%%%%%%%%%%%%%%%%%
%%%%%%%%%%%%%%%%%%%%%%%%%%%%%%%%%%%%%%%%%%%%%%%%%%%%%%%%%%%%%%%%%%%%%%%%%%%%
%%%%%%%%%%%%%%%%%%%%%%%%%%%%%%%%%%%%%%%%%%%%%%%%%%%%%%%%%%%%%%%%%%%%%%%%%%%%

\newpage
\pagenumbering{arabic}
\setcounter{page}{2}
{\huge{\textbf{Аппроксимация функции многочленом Лагранжа}}}\\
\LARGE
Даны два уравнения:
\[f_1(x)=\ctg(x) + x^2 \hspace*{1cm} f_2(x) = x^5 - 3.2x^3 + 2.5x^2 - 7x + 1.5\]
Были выбраны следующие интервалы непрерывности:
\begin{align}
    f_1\subset[0.5, 2.75] \hspace*{2cm}f_2\subset[-2.4, 2.1]\hspace*{5cm}
\end{align}
Рассмотрим алгоритм приближения функций по её известным табличным значениям на примере $f_1(x)$. Вычислим точно её значения в нескольких точках:

\begin{center}
    \begin{align}
        \begin{tabular}{|c|c|c|}
            \hline
            $i$ & $\mathbf{x_i}$ & $\mathbf{y_i}$ \\
            \hline
            0 & 0.5 & 2.08049 \\
            \hline
            1 & 1.25 & 1.89477 \\
            \hline
            2 & 2 & 3.54234 \\
            \hline
            3 & 2.75 & 5.14071 \\
            \hline
        \end{tabular}
    \end{align}
\end{center}
По формуле многочлена Лагранжа $(n=i_{\max} = 3)$:
\begin{align}
   & L(x)=\sum_{i=0}^{n}y_il_i(x),
\end{align}
где базисные полиномы определяются по формуле:
\begin{align}
    l_i(x)=\prod_{j=0,j\ne i}^{n}\cfrac{x-x_j}{x_i-x_j}
\end{align}

%%%%%%%%%%%%%%%%%%%%%%%%%%%%%%%%%%%%%%%%%%%%%%%%%%%%%%%%%%%%%%%%%%%%%%%%%%%%
%%%%%%%%%%%%%%%%%%%%%%%%%%%%%%%%%%%%%%%%%%%%%%%%%%%%%%%%%%%%%%%%%%%%%%%%%%%%
%%%%%%%%%%%%%%%%%%%%%%%%%%%%%%%%%%%%%%%%%%%%%%%%%%%%%%%%%%%%%%%%%%%%%%%%%%%%

{\huge{\textbf{Численный расчёт:}}}
\begin{align} % Глобальный Лагранж
    L(x) &=
       2.08049\cdot\frac{x-1.25}{0.5-1.25}\cdot\frac{x-2}{0.5-2}\cdot\frac{x-2.75}{0.5-2.75} +\\
    &+ 1.89477\cdot\frac{x-0.5}{1.25-0.5}\cdot\frac{x-2}{1.25-2}\cdot\frac{x-2.75}{1.25-2.75} +\\
    &+ 3.54234\cdot\frac{x-0.5}{2-0.5}\cdot\frac{x-1.25}{2-1.25}\cdot\frac{x-2.75}{2-2.75} +\\
    &+ 5.14071\cdot\frac{x-0.5}{2.75-0.5}\cdot\frac{x-1.25}{2.75-1.25}\cdot\frac{x-2}{2.75-2} =\\
    &= -0.821922 x^3 + 4.93153 x^2 - 9.40073 x + 5.65071 +\\
    &+ 2.24565 x^3 - 11.7897 x^2 + 17.6845 x - 6.17555 -\\
    &- 4.19833 x^3 + 18.8925 x^2 - 22.8284 x + 7.21588 +\\
    &+ 2.0309 x^3 - 7.61587 x^2 + 8.37745 x - 2.53862 = \\
    &\bf{=} 0.07822 x^3 + 4.41846 x^2 - 6.16718 x + 4.15242
\end{align}
Получено кубическое уравнение. Табулируя его по $4\cdot100$ точкам, получим следующую таблицу значений:

{\Large
\begin{center}
\begin{align}
    \begin{tabular}{|c|c|c|c|c|}
        \hline
        $k$ & $x_k$ & $f_1(x_k)$ & $L(x_k)$ & $|L(x_k)-f_1(x_k)|$ \\
        \hline
        1 & 0.5      & 2.08049 & 2.08049 & 0 \\
        \hline
        2 & 0.505639 & 2.061887& 2.06759 &  0.00571114 \\
        \hline
        3 & 0.511278 & 2.04382 & 2.05489 &  0.0110783 \\
        \hline
        ... & ... & ... & ... &  ... \\
        \hline
        399 & 2.74436   & 5.14792 & 5.13354 & 0.0143777 \\
        \hline
        400 & 2.75 & 5.14071  & 5.14071 &  0 \\
        \hline
    \end{tabular}\\
    \langle|L(x_k)-f_1(x_k)|\rangle=5.09055\cdot10^{-2}
\end{align}
\end{center}}
Далее построим график получившегося полинома.
\newpage

%%%%%%%%%%%%%%%%%%%%%%%%%%%%%%%%%%%%%%%%%%%%%%%%%%%%%%%%%%%%%%%%%%%%%%%%%%%%
%%%%%%%%%%%%%%%%%%%%%%%%%%%%%%%%%%%%%%%%%%%%%%%%%%%%%%%%%%%%%%%%%%%%%%%%%%%%
%%%%%%%%%%%%%%%%%%%%%%%%%%%%%%%%%%%%%%%%%%%%%%%%%%%%%%%%%%%%%%%%%%%%%%%%%%%%

{\huge{\bf{Графики:}}}

\vspace{-0.5cm}
{\normalsize Рис.1: График интерполяции $f_1(x)$ полиномом Лагранжа по 4 узлам, средняя ошибка  $\langle\varepsilon\rangle=5.09055\cdot10^{-2}$}

\begin{center}
    \includegraphics[width=0.88\linewidth]{pics/pic1}\\
    \vspace{-0.7cm}
    {\small В узлах значение полинома в точности совпадает со значением интерполируемой функции.}
\end{center}

\vspace{-0.5cm}
{\normalsize Рис.2: Та же функция, 9 узлов, средняя ошибка $\langle\varepsilon\rangle=3.61955\cdot10^{-3}$}
\begin{center}
    \includegraphics[width=0.88\linewidth]{pics/pic2}\\
    \vspace{-0.7cm}
    {\small С ростом числа узлов растёт точность, особенно в середине интервала.}
\end{center}

%%%%%%%%%%%%%%%%%%%%%%%%%%%%%%%%%%%%%%%%%%%%%%%%%%%%%%%%%%%%%%%%%%%%%%%%%%%%
%%%%%%%%%%%%%%%%%%%%%%%%%%%%%%%%%%%%%%%%%%%%%%%%%%%%%%%%%%%%%%%%%%%%%%%%%%%%
%%%%%%%%%%%%%%%%%%%%%%%%%%%%%%%%%%%%%%%%%%%%%%%%%%%%%%%%%%%%%%%%%%%%%%%%%%%%

{\normalsize Рис.3: На $30$ узлах середина отрезка достигает машинной точности. $\langle\varepsilon\rangle = 2.71830\cdot10^{-9}$}
\begin{center}
    \includegraphics[width=0.88\linewidth]{pics/pic3}\\
\end{center}

{\normalsize Рис.4: С дальнейшим ростом числа узлов всё заметнее проявляется феномен Рунге:
\begin{center}
    \includegraphics[width=0.88\linewidth]{pics/pic4}\\
    \vspace{-0.7cm}
\end{center}
Cредняя ошибка интерполяции теперь растёт: $\langle\varepsilon\rangle = 1.13967\cdot10^{-5}$}

\newpage
Теперь для $f_2(x)$:\\
{\normalsize Рис.5: $\langle\varepsilon\rangle = 3.8292$
\begin{center}
    \includegraphics[width=0.88\linewidth]{pics/pic5}\\
\end{center}}

{\normalsize Рис.6: При интерполяции на 6 узлах полином Лагранжа {\bf в точности} совпадает с многочленом 5-й степени.
    \begin{center}
        \includegraphics[width=0.88\linewidth]{pics/pic6}\\
    \end{center}
$\langle\varepsilon\rangle =1.91954\cdot10^{15}$, добавлять больше узлов не имеет смысла.
\newpage

%%%%%%%%%%%%%%%%%%%%%%%%%%%%%%%%%%%%%%%%%%%%%%%%%%%%%%%%%%%%%%%%%%%%%%%%%%%%
%%%%%%%%%%%%%%%%%%%%%%%%%%%%%%%%%%%%%%%%%%%%%%%%%%%%%%%%%%%%%%%%%%%%%%%%%%%%
%%%%%%%%%%%%%%%%%%%%%%%%%%%%%%%%%%%%%%%%%%%%%%%%%%%%%%%%%%%%%%%%%%%%%%%%%%%%

{\normalsize Рис.7: С ростом числа узлов интерполяции так же отметим проявление феномена Рунге:}
\begin{center}
    \includegraphics[width=0.88\linewidth]{pics/pic7}\\
\end{center}
\vspace{-0.5cm}
$\langle\varepsilon\rangle=5.17794\cdot10^{-11}$.\\

{\normalsize Рис.8: График зависимости средней ошибки аппроксимации полиномом Лагранжа от количества узлов интерполяции:}
\begin{center}
    \includegraphics[width=0.91\linewidth]{pics/pic8}\\
\end{center}
\newpage

%%%%%%%%%%%%%%%%%%%%%%%%%%%%%%%%%%%%%%%%%%%%%%%%%%%%%%%%%%%%%%%%%%%%%%%%%%%%
%%%%%%%%%%%%%%%%%%%%%%%%%%%%%%%%%%%%%%%%%%%%%%%%%%%%%%%%%%%%%%%%%%%%%%%%%%%%
%%%%%%%%%%%%%%%%%%%%%%%%%%%%%%%%%%%%%%%%%%%%%%%%%%%%%%%%%%%%%%%%%%%%%%%%%%%%

{\huge{\textbf{Аппроксимация функции сплайном Эрмита}}}\\
\LARGE Для построения сплайна Эрмита используем прежнюю таблицу узлов $(2)$ функции $f_1(x)$. Используя трёхточечный многочлен Лагранжа, рассчитаем значение для двух дополнительных точек, отстоящих от известных граничных значений на один интервал:

\begin{center}
    \begin{tabbing}
    \hspace*{3.5cm}\= \hspace*{3.5cm} \= \hspace*{4.8cm} \= \kill
    \>
    \huge$\begin{smallmatrix}
        \text{Узлы для}\\
        \text{нахождения}\\
        \text{точки слева}
    \end{smallmatrix}
    \begin{cases}
         \\\\
    \end{cases}$
    \>
    \begin{tabular}{|c|c|c|}
        \hline
        $i$ & $x_i$ & $y_i$ \\
        \hline
        -1 & -0.25 & {\bf4.0995} \\
        \hline
        0 & 0.5 & 2.08049 \\
        \hline
        1 & 1.25 & 1.89477 \\
        \hline
        2 & 2 & 3.54234 \\
        \hline
        3 & 2.75 & 5.14071 \\
        \hline
        4 & 3.5 & {\bf6.68988} \\
        \hline
    \end{tabular}
    \>\raisebox{-0.7\height}{
    \Large$\begin{rcases}
        \\\\\\
    \end{rcases}
    {\begin{smallmatrix}
        \text{Узлы для}\\
        \text{нахождения}\\
        \text{точки справа}
    \end{smallmatrix}}$\\
    }
    \end{tabbing}
\end{center}
Полученные лагранжевы многочлены:
\begin{align}
    &L_{-1}(x) = 1.62959x^2 - 3.09941x - 3.222798\\
    &L_{4}(x) = 1.071816x^2 - 2.123305x + 2.874188
\end{align}
Методом усреднения секущих
\begin{align}
    \mathtt{f'}(x_i) &= \frac{f(x_{i+1})-f(x_{i-1})}{x_{i+1}-x_{i-1}}
\end{align}
получим значения первой производной (фактически - тангенса секущей) в исходных точках $\{x_0..x_3\}$:
\begin{align}
    \mathtt{f_1'}(x_0) = \frac{f(x_1)-f(x_{-1})}{x_1-x_{-1}}=\frac{1.89477-4.0995}{1.25+0.25}=-1.46982\\
    \mathtt{f_1'}(x_1) = \frac{f(x_2)-f(x_0)}{x_2-x_0}=\frac{3.54234-2.08049}{2-0.5}=0.9745667
\end{align}
\begin{align}
    \mathtt{f_1'}(x_2) = \frac{f(x_3)-f(x_1)}{x_3-x_1}=\frac{5.14071-1.89477}{2.75-1.25}=2.16396 \\
    \mathtt{f_1'}(x_3) = \frac{f(x_4)-f(x_2)}{x_4-x_2}=\frac{6.68988-3.54234}{3.5-2}= 2.09836
\end{align}
Построим таблицу абсолютной ошибки производной, взятой методом секущих:
\begin{center}
    \begin{align}
        \begin{tabular}{|c|c|c|c|c|}
            \hline
            $i$ & $x_i$ & $f_1'(x_i)$ & $\mathtt{f_1'}(x_i)$ & $|\mathtt{f_1'}(x_i)-f_1'(x_i)|$ \\
            \hline
            0 & 0.5 & -3.35069 & -1.46981 & 1.88088\\
            \hline
            1 &  1.25 & 1.38959 & 0.97457 &0.415020\\
            \hline
            2 &  2 &  2.79055 & 2.16396 &0.626590\\
            \hline
            3 &  2.75 & -1.36506 & 2.09836 &3.46342\\
            \hline
        \end{tabular}\\
\varepsilon\in[0.42, 3.46]
    \end{align}
\end{center}
\vspace{-0.5cm}
    Теперь перейдём к построению сплайна. Формулы коэффициентов полинома Эрмита:
\begin{align}
    H_{00} &= 2 x_N^3 - 3 x_N^2 + 1\\
    H_{10} &= x_N^3 - 2 x_N^2 + x_N\\
    H_{01} &= -2 x_N^3 + 3x_N^2\\
    H_{11} &= x_N^3 - x_N^2&
\end{align}
Где $x_N$ - нормализованное значение аргумента $x$ на отрезке $[0, 1]$ между двумя узлами интерполяции $x_m,x_{m+1}$:
\begin{align}
    x_N = \frac{x - x_m}{x_{m+1} - x_{m}}
\end{align}
\newpage
{\huge{\textbf{Численный расчёт:}}}

В нашем примере между 4 узлами имеем 3 интервала интерполяции:
\begin{center}
    \begin{align}
        \begin{tabular}{|c|c|c|c|}
            \hline
             & $x_m$ & $x_{m+1}$ & $x_N$\\
            \hline
            1 &  0.5 & 1.25 & $\frac{4x-2}{3}=0.66667x$\\
            \hline
            2 &  1.25 &  2 & $\frac{4x-5}{3}=-0.33333x$\\
            \hline
            3 &  2 & 2.75 & $\frac{4x-8}{3}=-1.33333x$\\
            \hline
        \end{tabular}
    \end{align}
\end{center}
Общий член полинома Эрмита:
\begin{align}
    H(x) = H_{00}\cdot f(x_m) + H_{10}\cdot f'(x_m)\cdot(x_{m+1}-x_m) + \\H_{01}\cdot f(x_{m+1}) + H_{11}\cdot f'(x_{m+1})\cdot(x_{m+1}-x_m)
\end{align}
Наконец, подставляя сюда значения тангенсов секущих из $(23)$, вычислим значения $4\cdot100$ точек сплайна:
\begin{center} % Произвольная табуляция
\begin{align}
    \begin{tabular}{|c|c|c|c|c|}
        \hline
        $k$ & $x_k$ & $f_1(x_k)$ & $H(x_k)$ & $|H(x_k)-f_1(x_k)|$ \\
        \hline
        1 & 0.5      & 2.08049 & 2.08049 & 0 \\
        \hline
        2 & 0.505639 & 2.061887& 2.07904 &  0.0171649 \\
        \hline
        3 & 0.511278 & 2.04382 & 2.07749 &  0.0336751 \\
        \hline
        ... & ... & ... & ... &  ... \\
        \hline
        399 & 2.74436   & 5.14792 &	5.12888 &  0.0190426 \\
        \hline
        400 & 2.75 & 5.14071  & 5.14071 &  $6.03961\cdot10^{-14}$\\
        \hline
    \end{tabular}\\
    \langle|H(x_k)-f_1(x_k)|\rangle=1.13928\cdot10^{-1}
\end{align}
\end{center}
\newpage

{\huge{\textbf{Графики:}}}\\[-0.5cm]
{\normalsize Рис.9: График интерполяции $f_1(x)$ сплайном Эрмита по 4 узлам, средняя ошибка  $\langle\varepsilon\rangle=1.13928\cdot10^{-1}$}
\begin{center}
    \includegraphics[width=0.9\linewidth]{pics/pic9}\\
    \vspace{-0.7cm}
    {\small На среднем отрезке сплайн пересекает график функции и ошибка устремляется к $0$.}
\end{center}

\vspace{-0.5cm}
{\normalsize Рис.10: Та же функция, 9 узлов, средняя ошибка $\langle\varepsilon\rangle=1.09601\cdot10^{-2}$}
\begin{center}
    \includegraphics[width=0.9\linewidth]{pics/pic10}\\
    \vspace{-0.7cm}
    {\small Теперь почти на каждом междоузлии есть сплайн пересекает график функции.}
\end{center}

%%%%%%%%%%%%%%%%%%%%%%%%%%%%%%%%%%%%%%%%%%%%%%%%%%%%%%%%%%%%%%%%%%%%%%%%%%%%
%%%%%%%%%%%%%%%%%%%%%%%%%%%%%%%%%%%%%%%%%%%%%%%%%%%%%%%%%%%%%%%%%%%%%%%%%%%%
%%%%%%%%%%%%%%%%%%%%%%%%%%%%%%%%%%%%%%%%%%%%%%%%%%%%%%%%%%%%%%%%%%%%%%%%%%%%

\newpage
{\normalsize Рис.11: Ошибка убывает не так быстро, как у полинома Лагранжа $\langle\varepsilon\rangle=3.61955\cdot10^{-3}$}
\begin{center}
    \includegraphics[width=0.9\linewidth]{pics/pic11}\\
    \vspace{-0.7cm}
    {\small Но распределяется, в среднем, более равномерно, чем у него}
\end{center}
\vspace{-0.7cm}
{\normalsize Рис.12: В отличие от глобальной интерполяции Лагранжа, сплайн в меньшей степени подвержен феномену Рунге.}
\begin{center}
    \includegraphics[width=0.9\linewidth]{pics/pic12}
    \vspace{-0.7cm}
\end{center}

{\small $\langle\varepsilon\rangle=3.76721\cdot10^{-5}$}

%%%%%%%%%%%%%%%%%%%%%%%%%%%%%%%%%%%%%%%%%%%%%%%%%%%%%%%%%%%%%%%%%%%%%%%%%%%%
%%%%%%%%%%%%%%%%%%%%%%%%%%%%%%%%%%%%%%%%%%%%%%%%%%%%%%%%%%%%%%%%%%%%%%%%%%%%
%%%%%%%%%%%%%%%%%%%%%%%%%%%%%%%%%%%%%%%%%%%%%%%%%%%%%%%%%%%%%%%%%%%%%%%%%%%%

\newpage
Теперь для $f_2(x)$:\\[-0.5cm]
{\normalsize Рис.13: $\langle\varepsilon\rangle = 5.374829$}
\begin{center}
        \includegraphics[width=0.9\linewidth]{pics/pic13}\\
\end{center}
\vspace{-0.7cm}
{\normalsize Рис.14: $\langle\varepsilon\rangle = 0.441794$}
\begin{center}
    \includegraphics[width=0.88\linewidth]{pics/pic14}\\
     \vspace{-0.7cm}
      {\small В отличие от метода Лагранжа, сплайн не совпадает с полиномиальной функцией, даже когда количество $n>5$.}
\end{center}
\newpage

%%%%%%%%%%%%%%%%%%%%%%%%%%%%%%%%%%%%%%%%%%%%%%%%%%%%%%%%%%%%%%%%%%%%%%%%%%%%
%%%%%%%%%%%%%%%%%%%%%%%%%%%%%%%%%%%%%%%%%%%%%%%%%%%%%%%%%%%%%%%%%%%%%%%%%%%%
%%%%%%%%%%%%%%%%%%%%%%%%%%%%%%%%%%%%%%%%%%%%%%%%%%%%%%%%%%%%%%%%%%%%%%%%%%%%

{\normalsize Рис.15: $\langle\varepsilon\rangle = 6.6875\cdot10^{-2}$}
\begin{center}
    \includegraphics[width=0.9\linewidth]{pics/pic15}\\
    \vspace{-0.7cm}
\end{center}

{\normalsize Рис.16: Сводный график зависимости ошибки от количества узлов:}
\begin{center}
    \includegraphics[width=0.91\linewidth]{pics/pic16}\\
\end{center}
\vspace{-1cm}
{\small Средняя ошибка интерполяции сплайном уменьшается при увеличении количества узлов, но не так быстро. В то же время,\\[-0.5cm]
ошибка интерполяции полиномом Лагранжа сначала снижается (при интерполяции полинома -- до уровня машинной точности),\\[-0.5cm]
однако затем начинает возрастать из-за феномена Рунге.}

%%%%%%%%%%%%%%%%%%%%%%%%%%%%%%%%%%%%%%%%%%%%%%%%%%%%%%%%%%%%%%%%%%%%%%%%%%%%
%%%%%%%%%%%%%%%%%%%%%%%%%%%%%%%%%%%%%%%%%%%%%%%%%%%%%%%%%%%%%%%%%%%%%%%%%%%%
%%%%%%%%%%%%%%%%%%%%%%%%%%%%%%%%%%%%%%%%%%%%%%%%%%%%%%%%%%%%%%%%%%%%%%%%%%%%

\newpage
{\huge{\textbf{Исследование сходимости полинома Лагранжа на равномерной и чебышёвской решётках}}}\\
Решётка Чебышёва -- это корни многочлена Чебышёва первого рода, которые используются при полиномиальной интерполяции для снижения влияния феномена Рунге.

{\normalsize Рис.17:}
\begin{center}
    \includegraphics[width=0.8\textwidth]{pics/chebGrid.png}
\end{center}

Для натурального желаемого числа $n$ узлы Чебышёва на отрезке $[-1, 1]$ задаются формулой:
\begin{align}
     x_k = \cos\left(\frac{2k - 1}{2n}\pi\right), &\hspace{1cm} k=0,...,n-1
\end{align}

Для произвольного отрезка $[a, b]$, применяя афинное преобразование, имеем:
\begin{align}
    x_k = \left(\frac{a+b}{2}\right) + \left(\frac{b-a}{2}\right) \cos\left(\frac{2k-1}{2n}\pi\right), &\hspace{1cm} k=0,...,n-1
\end{align}
что и было использовано в программной реализации построения решётки для интерполяции.

%%%%%%%%%%%%%%%%%%%%%%%%%%%%%%%%%%%%%%%%%%%%%%%%%%%%%%%%%%%%%%%%%%%%%%%%%%%%
%%%%%%%%%%%%%%%%%%%%%%%%%%%%%%%%%%%%%%%%%%%%%%%%%%%%%%%%%%%%%%%%%%%%%%%%%%%%
%%%%%%%%%%%%%%%%%%%%%%%%%%%%%%%%%%%%%%%%%%%%%%%%%%%%%%%%%%%%%%%%%%%%%%%%%%%%

\newpage
{\huge{\bf{Графики:}}}

\vspace{-0.5cm}
{\normalsize Рис.18: Сравнение интерполяции $f_1(x)$ полиномом Лагранжа на равномерной решётке и решётке Чебышёва}
\begin{center}
    \includegraphics[width=0.9\linewidth]{pics/pic18}\\
    \vspace{-0.7cm}
    {\small
        $\langle\varepsilon_U\rangle=5.09055\cdot10^{-2}$ \hspace{3cm}
        $\langle\varepsilon_T\rangle=2.87312\cdot10^{-2}$
    }
\end{center}
{\normalsize Рис.19:}
\begin{center}
    \includegraphics[width=0.88\linewidth]{pics/pic19}\\
    \vspace{-0.7cm}
    {\small
        $\langle\varepsilon_U\rangle=4.0589\cdot10^{-2}$ \hspace{3cm}
        $\langle\varepsilon_T\rangle=2.85518\cdot10^{-2}$
    }
\end{center}

%%%%%%%%%%%%%%%%%%%%%%%%%%%%%%%%%%%%%%%%%%%%%%%%%%%%%%%%%%%%%%%%%%%%%%%%%%%%
%%%%%%%%%%%%%%%%%%%%%%%%%%%%%%%%%%%%%%%%%%%%%%%%%%%%%%%%%%%%%%%%%%%%%%%%%%%%
%%%%%%%%%%%%%%%%%%%%%%%%%%%%%%%%%%%%%%%%%%%%%%%%%%%%%%%%%%%%%%%%%%%%%%%%%%%%

\newpage
{\normalsize Рис.20: Становится заметно, насколько профиль ошибки на узлах Чебышёва более ровный}
\begin{center}
    \includegraphics[width=0.9\linewidth]{pics/pic20}\\
    \vspace{-0.7cm}
    {\small
        $\langle\varepsilon_U\rangle=8.20803\cdot10^{-5}$ \hspace{3cm}
        $\langle\varepsilon_T\rangle=3.50966\cdot10^{-5}$
    }
\end{center}
{\normalsize Рис.21: Несмотря на лучшую точность равномерной решётки в середине интервала, решётка Чебышёва нивелирует влияние феномена Рунге и в среднем выигрывает.}
\begin{center}
    \includegraphics[width=0.88\linewidth]{pics/pic21}\\
    \vspace{-0.7cm}
    {\small
        $\langle\varepsilon_U\rangle=7.08864\cdot10^{-10}$ \hspace{3cm}
        $\langle\varepsilon_T\rangle=5.06756\cdot10^{-13}$
    }
\end{center}

%%%%%%%%%%%%%%%%%%%%%%%%%%%%%%%%%%%%%%%%%%%%%%%%%%%%%%%%%%%%%%%%%%%%%%%%%%%%
%%%%%%%%%%%%%%%%%%%%%%%%%%%%%%%%%%%%%%%%%%%%%%%%%%%%%%%%%%%%%%%%%%%%%%%%%%%%
%%%%%%%%%%%%%%%%%%%%%%%%%%%%%%%%%%%%%%%%%%%%%%%%%%%%%%%%%%%%%%%%%%%%%%%%%%%%

\newpage
Теперь для $f_2(x)$:\\[-0.5cm]
{\normalsize Рис.22: Сравнение интерполяции $f_2(x)$ полиномом Лагранжа по равномерной решётке и решётке Чебышёва}
\begin{center}
    \includegraphics[width=0.9\linewidth]{pics/pic22}\\
    \vspace{-0.7cm}
    {\small
        $\langle\varepsilon_U\rangle=3.8292$ \hspace{3cm}
        $\langle\varepsilon_T\rangle=2.55224$
    }
\end{center}
{\normalsize Рис.23:}
\begin{center}
    \includegraphics[width=0.9\linewidth]{pics/pic23}\\
    \vspace{-0.7cm}
    {\small
        $\langle\varepsilon_U\rangle=2.84741$ \hspace{3cm}
        $\langle\varepsilon_T\rangle=2.28239$
    }
\end{center}

%%%%%%%%%%%%%%%%%%%%%%%%%%%%%%%%%%%%%%%%%%%%%%%%%%%%%%%%%%%%%%%%%%%%%%%%%%%%
%%%%%%%%%%%%%%%%%%%%%%%%%%%%%%%%%%%%%%%%%%%%%%%%%%%%%%%%%%%%%%%%%%%%%%%%%%%%
%%%%%%%%%%%%%%%%%%%%%%%%%%%%%%%%%%%%%%%%%%%%%%%%%%%%%%%%%%%%%%%%%%%%%%%%%%%%

\newpage
{\normalsize Рис.24: На шести узлах интерполирующий полином опять совпадает с полиномиальной $f_2$:}
\begin{center}
    \includegraphics[width=0.9\linewidth]{pics/pic24}\\
    \vspace{-0.7cm}
    {\small
        $\langle\varepsilon_U\rangle=1.91954\cdot10^{-15}$ \hspace{3cm}
        $\langle\varepsilon_T\rangle=2.12541\cdot10^{-15}$
    }
\end{center}
{\normalsize Рис.25: С ростом числа узлов, решётка Чебышёва снова оказывается точнее в среднем.}
\begin{center}
    \includegraphics[width=0.88\linewidth]{pics/pic25}\\
    \vspace{-0.7cm}
    {\small
        $\langle\varepsilon_U\rangle=1.06776\cdot10^{-9}$ \hspace{3cm}
        $\langle\varepsilon_T\rangle=4.41698\cdot10^{-15}$
    }
\end{center}

%%%%%%%%%%%%%%%%%%%%%%%%%%%%%%%%%%%%%%%%%%%%%%%%%%%%%%%%%%%%%%%%%%%%%%%%%%%%
%%%%%%%%%%%%%%%%%%%%%%%%%%%%%%%%%%%%%%%%%%%%%%%%%%%%%%%%%%%%%%%%%%%%%%%%%%%%
%%%%%%%%%%%%%%%%%%%%%%%%%%%%%%%%%%%%%%%%%%%%%%%%%%%%%%%%%%%%%%%%%%%%%%%%%%%%

\newpage
{\normalsize Рис.26: Сводный график зависимости ошибки от количества узлов для разных решёток:}
\begin{center}
    \includegraphics[width=1\linewidth]{pics/pic26}
\end{center}
Вычислительный эксперимент показал, что решётка Чебышёва эффективно решает проблему появления феномена Рунге при глобальной полиномиальной интерполяции как для полиномиальной, так и для трансцендентной функции. Машинная точность была достигнута при, соответственно, $n_\text{узлов} > \text{\normalsize старшая степень полинома} $ и $n_\text{узлов} > 40$.

%%%%%%%%%%%%%%%%%%%%%%%%%%%%%%%%%%%%%%%%%%%%%%%%%%%%%%%%%%%%%%%%%%%%%%%%%%%%
%%%%%%%%%%%%%%%%%%%%%%%%%%%%%%%%%%%%%%%%%%%%%%%%%%%%%%%%%%%%%%%%%%%%%%%%%%%%
%%%%%%%%%%%%%%%%%%%%%%%%%%%%%%%%%%%%%%%%%%%%%%%%%%%%%%%%%%%%%%%%%%%%%%%%%%%%

\newpage
{\huge{\textbf{Исследование сходимости полинома Лагранжа и сплайна Эрмита при возмущении входных данных}}}\\
Формализуем понятие "возмущение". Допустим, что при измерении некоторой зависимости присутствовала случайная погрешность. Величина такой погрешности $\Delta_{y_i}$ -- случайная величина, подчиняющаяся, в большинстве случаев, нормальному распределению:
\begin{align}
    \rho(y) = \frac{1}{\sigma\sqrt{2\pi}}\exp\left[-\frac{1}{2}\left(\frac{y-\mu}{\sigma}\right)^2\right].
\end{align}
В нашем случае $\mu=0$, стандартное отклонение $\sigma$ примем пропорциональным амплитуде значений исследуемой функции $f$ на выбранном участке непрерывности $[a,b], (1)$:
\begin{align}
    \sigma = \mathbf{K} \frac{\sup f(x) - \inf f(x)}{2},
    \hspace{1cm} x\in[a,b]
\end{align}
Так, для $f_1$, интерполированной по четырём узлам $(2)$:
\begin{align}
    \sigma_{f_1} = \mathbf{K} \frac{5.15071 - 1.89477}{2} = 1.62797\mathbf{K}
\end{align}
В то же время, для $f_2$, интерполированной по четырём узлам:
\begin{align}
    \sigma_{f_2} = \mathbf{K} \frac{11.5673 - (-2.41344)}{2} = 6.99037\mathbf{K}
\end{align}
Видим, что для разных функций, константа перед $\mathbf{K}$ оказывается разной, т.к. на исследуемых отрезках у этих функций различаются амплитуды. Коэффициент $\mathbf{K}$ введён, чтобы свести исследование различных зависимостей к простому вопросу: как соотносится величина шума с амплитудой функции на исследуемом отрезке. Считаю разумным изучать поведение средней точности интерполяции при $\mathbf{K} \lesssim 30\%$.

%%%%%%%%%%%%%%%%%%%%%%%%%%%%%%%%%%%%%%%%%%%%%%%%%%%%%%%%%%%%%%%%%%%%%%%%%%%%
%%%%%%%%%%%%%%%%%%%%%%%%%%%%%%%%%%%%%%%%%%%%%%%%%%%%%%%%%%%%%%%%%%%%%%%%%%%%
%%%%%%%%%%%%%%%%%%%%%%%%%%%%%%%%%%%%%%%%%%%%%%%%%%%%%%%%%%%%%%%%%%%%%%%%%%%%

\newpage
{\huge{\bf{Примеры возмущения входных данных:}}}\\
{\normalsize Рис.27: Одна из возможных конфигураций узлов $f_1(x)$ с добавленным шумом.}
\begin{center}
    \includegraphics[width=0.9\linewidth]{pics/pic27}
\end{center}
{\normalsize Рис.28: Одна из возможных конфигураций узлов $f_2(x)$ с добавленным шумом.}
\begin{center}
    \includegraphics[width=0.9\linewidth]{pics/pic28}
    \vspace{-0.7cm}
\end{center}

%%%%%%%%%%%%%%%%%%%%%%%%%%%%%%%%%%%%%%%%%%%%%%%%%%%%%%%%%%%%%%%%%%%%%%%%%%%%
%%%%%%%%%%%%%%%%%%%%%%%%%%%%%%%%%%%%%%%%%%%%%%%%%%%%%%%%%%%%%%%%%%%%%%%%%%%%
%%%%%%%%%%%%%%%%%%%%%%%%%%%%%%%%%%%%%%%%%%%%%%%%%%%%%%%%%%%%%%%%%%%%%%%%%%%%

\newpage

Вместо рисования перегруженных деталями графиков ниже приведён сводный график всех испытаний с усреднением по $\mathbf{100}$ повторениям.

{\normalsize Рис.29: Зависимость средней абсолютной ошибки интерполяции $f_1$ и $f_2$ на $\mathbf{20}$ узлах от $\mathbf{K}\in[10^{-6}, 0.3]$.}
\begin{center}
    \includegraphics[width=0.9\linewidth]{pics/pic29}\\
\end{center}
\vspace{-0.5cm}
Основываясь на рис. 29 можно заключить:
\begin{itemize}
    \large
    \item При наличии возмущений в узлах полиномиальная функция $f_2$ (красная палитра) при прочих равных даёт, в среднем, б\'{о}льшую абсолютную ошибку, чем трансцендентная функция $f_1$ (синяя палитра);
    \item Если величина возмущений в узлах распределена нормально, то средняя абсолютная ошибка интерполяции полиномом Лагранжа ($L_U, L_T$) линейно зависит от величины стандартного отклонения распределения ошибки $\Delta_{y_i}$ ($\sigma$);
    \item Использование решётки Чебышёва ($L_T$) снижает ошибку интерполяции Лагранжа на $\sim2$ порядка по сравнению с равномерной решёткой;
    \item В случае, если $\sigma$ распределения $\Delta_{y_i}$ превышает $\sim0.2\%$ от амплитуды значений функции на отрезке, сплайн Эрмита ($H_U$), так же как и решётка Чебышёва, снижает среднюю абсолютную ошибку на 2 порядка, но если $\sigma$ распределения $\Delta_{y_i}$ мала ($<0.04\%$), то сплайн не даёт никакого преимущества перед неравномерной решёткой, а в худших случаях, (очень низкий уровень шума) может давать ошибку, превышающую такую в равномерной решётке.
\end{itemize}
\end{document}