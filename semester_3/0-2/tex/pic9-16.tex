\begin{figure}[H]
    \centering
    \caption{График интерполяции $f_1(x)$ сплайном Эрмита по 4 узлам, $\varepsilon \le 0.298639$}
    \includegraphics[width=0.8\linewidth]{pics/pic9}
    \label{pic:9}
\end{figure}
На среднем отрезке сплайн пересекает график функции и ошибка устремляется к $0$.


\vspace{-0.5cm}
{\normalsize Рис.10: Та же функция, 9 узлов, средняя ошибка $\langle\varepsilon\rangle=1.09601\cdot10^{-2}$}
\begin{center}
    \includegraphics[width=0.9\linewidth]{pics/pic10}\\
    \vspace{-0.7cm}
    {\small Теперь почти на каждом междоузлии есть сплайн пересекает график функции.}
\end{center}

%%%%%%%%%%%%%%%%%%%%%%%%%%%%%%%%%%%%%%%%%%%%%%%%%%%%%%%%%%%%%%%%%%%%%%%%%%%%
%%%%%%%%%%%%%%%%%%%%%%%%%%%%%%%%%%%%%%%%%%%%%%%%%%%%%%%%%%%%%%%%%%%%%%%%%%%%
%%%%%%%%%%%%%%%%%%%%%%%%%%%%%%%%%%%%%%%%%%%%%%%%%%%%%%%%%%%%%%%%%%%%%%%%%%%%

\newpage
{\normalsize Рис.11: Ошибка убывает не так быстро, как у полинома Лагранжа $\langle\varepsilon\rangle=3.61955\cdot10^{-3}$}
\begin{center}
    \includegraphics[width=0.9\linewidth]{pics/pic11}\\
    \vspace{-0.7cm}
    {\small Но распределяется, в среднем, более равномерно, чем у него}
\end{center}
\vspace{-0.7cm}
{\normalsize Рис.12: В отличие от глобальной интерполяции Лагранжа, сплайн в меньшей степени подвержен феномену Рунге.}
\begin{center}
    \includegraphics[width=0.9\linewidth]{pics/pic12}
    \vspace{-0.7cm}
\end{center}

{\small $\langle\varepsilon\rangle=3.76721\cdot10^{-5}$}

%%%%%%%%%%%%%%%%%%%%%%%%%%%%%%%%%%%%%%%%%%%%%%%%%%%%%%%%%%%%%%%%%%%%%%%%%%%%
%%%%%%%%%%%%%%%%%%%%%%%%%%%%%%%%%%%%%%%%%%%%%%%%%%%%%%%%%%%%%%%%%%%%%%%%%%%%
%%%%%%%%%%%%%%%%%%%%%%%%%%%%%%%%%%%%%%%%%%%%%%%%%%%%%%%%%%%%%%%%%%%%%%%%%%%%

\newpage
Теперь для $f_2(x)$:\\[-0.5cm]
{\normalsize Рис.13: $\langle\varepsilon\rangle = 5.374829$}
\begin{center}
    \includegraphics[width=0.9\linewidth]{pics/pic13}\\
\end{center}
\vspace{-0.7cm}
{\normalsize Рис.14: $\langle\varepsilon\rangle = 0.441794$}
\begin{center}
    \includegraphics[width=0.88\linewidth]{pics/pic14}\\
    \vspace{-0.7cm}
    {\small В отличие от метода Лагранжа, сплайн не совпадает с полиномиальной функцией, даже когда количество $n>5$.}
\end{center}
\newpage

%%%%%%%%%%%%%%%%%%%%%%%%%%%%%%%%%%%%%%%%%%%%%%%%%%%%%%%%%%%%%%%%%%%%%%%%%%%%
%%%%%%%%%%%%%%%%%%%%%%%%%%%%%%%%%%%%%%%%%%%%%%%%%%%%%%%%%%%%%%%%%%%%%%%%%%%%
%%%%%%%%%%%%%%%%%%%%%%%%%%%%%%%%%%%%%%%%%%%%%%%%%%%%%%%%%%%%%%%%%%%%%%%%%%%%

{\normalsize Рис.15: $\langle\varepsilon\rangle = 6.6875\cdot10^{-2}$}
\begin{center}
    \includegraphics[width=0.9\linewidth]{pics/pic15}\\
\end{center}

{\normalsize Рис.16: Сводный график зависимости ошибки от количества узлов:}
\begin{center}
    \includegraphics[width=0.91\linewidth]{pics/pic16}\\
\end{center}
{\small Средняя ошибка интерполяции сплайном уменьшается при увеличении количества узлов, но не так быстро. В то же время,\\[-0.5cm]
    ошибка интерполяции полиномом Лагранжа сначала снижается (при интерполяции полинома -- до уровня машинной точности),\\[-0.5cm]
    однако затем начинает возрастать из-за феномена Рунге.}