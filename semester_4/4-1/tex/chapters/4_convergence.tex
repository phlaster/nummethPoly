\section{Анализ сходимости метода}

\subsection{Графики}
\begin{figure}[H]
    \centering
    \caption{Графики точного и приближенного решений, графики ошибки на отрезке}
    \includegraphics[width=0.95\linewidth]{pics/1}
    \label{fig:1}
\end{figure}
\begin{figure}[H]
    \centering
    \caption{Зависимость ошибки от шага}
    \includegraphics[width=0.95\linewidth]{pics/2}
    \label{fig:1}
\end{figure}
\begin{figure}[H]
    \centering
    \caption{Зависимость числа разбиений и наибольшей абсолютной ошибки от заданной точности}
    \includegraphics[width=0.95\linewidth]{pics/3-4}
    \label{fig:1}
\end{figure}

\subsection{Комментарии}

\begin{enumerate}
    \item На первом рисунке показан профиль ошибки решения д.у. на заданном отрезке. Ошибка растёт с удалением от точки $x_0$. С измельчением шага вдвое средняя ошибка убывает примерно вчетверо, что подтверждает теоретически обоснованный порядок сходимости 2.

    \item На втором рисунке показано как максимальная ошибка на интервале убывает вместе с квадратом шага разбиения. Отклонение от этой зависимости объясняется накоплением ошибки машинной точности при выполнении всё большего количества арифметических операций.

    \item На третьем рисунке показаны зависимости для решения, уточняемого по правилу Рунге и с адаптивным разбиением.  В первом случае (красный график) видны ступеньки, т.к. при недостаточной точности на шаге метод просто удваивает количество точек, чего хватает с избытком для шага, следующего за ним. Адаптивный же метод удваивает точность только локально на отрезках, на которых не выполняется условие Рунге, поэтому график более плавный, хотя точность решения не достигается.
\end{enumerate}