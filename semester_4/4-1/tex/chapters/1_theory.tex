\section{Теория}

\subsection{Определения}

Будем рассматривать обыкновенное дифференциальное уравнение (ОДУ) первого порядка:
\begin{equation}
    y'=f(x,y), ~~~ x \in [x_0, b]
\end{equation}
с начальным условием:
\begin{equation}
    y(x_0) = y_0,
\end{equation}
где $f(x,y)$ -- некоторая заданная, в общем случае, нелинейная функция двух переменных. Будем считать, что для данной задачи (1)--(2), называемой начальной задачей или задачей Коши, на отрезке $[x_0, b]$ существует её единственное решение $y = y(x)$, которое может быть найдено с помощью численных методов в виде числовой таблицы приближенных значений $y_i$ искомого решения $y(x)$ на некоторой сетке $x_i \in [x_0, b]$ значений аргумента $x$.

Разобьём $[x_0, b]$ на $n$ отрезков и будем считать, что вычисления проводятся с расчётным шагом $h=\cfrac{b-x_0}{n}$, то есть расчётными узлами служат точки $x_i=x_0 + ih|_{i=0,1,\ldots n}$. Идея построения явных методов Рунге-Кутты $p$-го порядка заключается в получении приближений к значениям $f(x_{i+1})$ по формуле вида:
\begin{equation}
    y_{i+1} = y_i + h\varphi(x_i, y_i, h),
\end{equation}
где $\varphi(x,y,h)$ -- некоторая функция, приближающая отрезок рада Тейлора до $p$-го порядка и не содержащая частных производных $f(x,y)$. Так, полагая в $\varphi(x,y,h) \equiv f(x,y)$ приходим к методу Эйлера как частному случаю метода Рунге-Кутты для $p=1$.

Для построения методов Рунге-Кутты порядка, выше первого, функцию $\varphi(x,y,h)$ берут многопараметрической, и подбирают ее параметры сравнением выражения (3) с многочленом Тейлора для $y(x)$ соответствующей желаемому порядку степени.

Для случая $p=2$ положим:
\begin{equation}
    \varphi(x,y,h) := c_1\cdot f(x,y) ~+~ c_2\cdot f(x+ah,~~ y+bh\cdot f(x,y))
\end{equation}
тогда, чтобы (3), которая, в таком случае, примет вид:
\begin{equation}
    y_{i+1} = y_i + h \cdot \left[ c_1\cdot f(x_i,y_i) ~+~ c_2\cdot f(x_i+ah,~~ y_i+bh\cdot f(x_i,y_i)) \right],
\end{equation}
определяла метод 2-го порядка (то есть максимальная локальная ошибка составила величину $O(h^3)$), параметры $c_1, c_2, a, b$ должны оказаться подчинены совокупности условий:
\begin{equation}
\begin{cases}
    c_1 + c_2 &= 1,\\
    c_2a &= 0.5,\\
    c_2b &= 0.5.
\end{cases}
\end{equation}
Полученная система условий содержит три уравнения относительно четырех параметров метода, что означает единственный свободный параметр. Положим $c_2 = \alpha (\ne 0)$, тогда из (6) получим:
\begin{equation}
    \begin{matrix}
        c_1=1-\alpha, & a=\dfrac{1}{2\alpha}, & b=\dfrac{1}{2\alpha},
    \end{matrix}
\end{equation}
подставляя которые в (5) получим общий вид семейства методов Рунге-Кутты 2-го порядка:
\begin{equation}
     y_{i+1} = y_i + h \cdot \left[ (1-\alpha)\cdot f(x_i,y_i) ~+~ \alpha f\left(x_i+\dfrac{h}{2\alpha},~~ y_i+\dfrac{h}{2\alpha}f(x_i,y_i)\right) \right],
\end{equation}
из которого нас интересует случай $\alpha=1$:
\begin{equation}
    y_{i+1} = y_i + hf\left(x_i+\dfrac{h}{2},~~ y_i+\dfrac{h}{2}f(x_i,y_i)\right),
\end{equation}
который называют методом средней точки.

\subsection{Алгоритм метода для фиксированного шага}
\begin{enumerate}
    \item Задать начальное приближение $y_0=y(x_0)$, рассчитать количество шагов разбиения $n:=\left\lceil\cfrac{b-x_0}{h}\right\rceil$;

    \item Для каждого $i=0,1,\ldots,n:$
    \begin{enumerate}
        \item Рассчитать $\eta^{(i)} := f\left(x_i+\dfrac{h}{2},~~ y_i+\dfrac{h}{2}f(x_i,y_i)\right)$

        \item Вычислить шаговую поправку: $\Delta y_i := h \eta^{(i)}$

        \item Получить значение в точке $x_{i+1} = x_i+h$ :~~~~ $y_{i+1} = y_i + \Delta y_i$
    \end{enumerate}
\end{enumerate}

\subsection{Алгоритм с использованием критерия Рунге}

\begin{enumerate}
    \item Выбрать необходимую точность $\varepsilon$, положить $n:=1$;

    \item По алгоритму 1.2 рассчитать приближение для $h_n=\frac{b-a}{n}$, где $y_i|_{i=0,1,\ldots,n}$;

    \item По алгоритму 1.2 рассчитать приближение для $h_{2n}=\frac{b-a}{2n}$, где $y_j|_{j=0,1,\ldots,2n}$;

    \item Найти максимальную разницу в общих узлах: $\Delta_{\max} = \max(\{|y_i-y_j|~:~j=2i,~i=0,1,\ldots,n\})$;

    \item Воспользоваться принципом Рунге c коэффициентом $\Theta=\dfrac{1}{2^p-1}$ для $p=2$:
    Если $\frac{1}{3} \Delta_{\max} \le \varepsilon \implies \blacksquare$

    \item Удвоить число узлов $n:= 2n$, вернуться к 2.
\end{enumerate}

\subsection{Алгоритм адаптивного шага}

\begin{enumerate}
    \item Задать первую точку решения $(x_0,~y_0)$ и начальную величину шага $h:=h_0$, желаемую точность $\varepsilon$;

    \item Рассчитать минимальное количество отрезков: $n_{\min} = \left\lceil \dfrac{b-a}{h_0} \right\rceil + 1$;


    \item В цикле по $i=1,\ldots,n$:
    \begin{enumerate}
        \item Границы отрезка: $(a_i, b_i) = (a+h_0(i-1), a+h_0i)$
        \item В бесконечном цикле, пока не будет выполнено условие Рунге:
        \begin{enumerate}
            \item На отрезке $(a_i, b_i)$ рассчитать решение $y_{j}$ для шага $h$ по алгоритму 1.2;

            \item На отрезке $(a_i, b_i)$ рассчитать решение $y_{2j}$ для шага $\frac{h}{2}$ по алгоритму 1.2;

            \item Найти максимальную разницу в общих узлах: $\Delta_{\max} = \max(\{|y_i-y_j|~:~j=2i,~i=0,1,\ldots,n\})$;

            \item Условие Рунге выхода из цикла 3.2: $\frac{1}{3} \Delta_{\max} \le \varepsilon ~~ \implies$ перейти к 3.3;

            \item Иначе уменьшить шаг разбиения отрезка: $h:=h/2$, продолжить цикл 4.2;
        \end{enumerate}

        \item Вернуться к начальной величине шага $h:=h_0$;
        \item К следующей итерации цикла 3.
    \end{enumerate}

    \item Общее численное решение ДУ на отрезке $[a,b]$ складывается из точек решений на сетках различной плотности полученных на каждом шаге 3.2.2 перед выполнением условия Рунге.


\end{enumerate}