\documentclass[a4paper]{article}
\usepackage[russian]{babel}
\usepackage[utf8]{inputenc}
\usepackage{setspace}
\setstretch{1.5}
\usepackage{anyfontsize}
\usepackage{amsmath, amsfonts, amssymb, amsthm, mathtools}
\usepackage{geometry}
\usepackage{graphicx}
\usepackage{wrapfig}
\usepackage{xcolor}
\usepackage{multicol}
\usepackage{listings}
\geometry{a4paper, portrait, margin=10mm, bmargin=15mm, tmargin=15mm}
\begin{document}
\include{modules/titlepage}
\clearpage
\pagenumbering{arabic}
\setcounter{page}{2}
\tableofcontents

\clearpage
\section{Теория}
\subsection{Определения}

Существует класс методов итерационного решения СЛАУ, называемый {\bf методами вариационного типа}. Их обоснование производится с привлечением теории оптимизации, т.к. решение системы сводится к решению эквивалентной экстремальной задачи. Пусть:
\begin{equation}
    \mathbf{Ax=b}
\end{equation}
нормальная n-мерная система, т.е. $\mathbf{A}$ -- положительно определённая симметричная матрица и пусть $(\cdot, \cdot)$ -- скалярное произведение в пространстве $\mathbb{R}^n$. Образуем квадратичный функционал:
\begin{equation}
    \mathbf{\Phi}(\mathbf{x})=(\mathbf{Ax}, \mathbf{x}) - 2(\mathbf{b}, \mathbf{x}) + c,
    ~~\text{ где }~~ c \in \mathbb{R}^1 ~~\text{ (произвольная постоянная)}.
\end{equation}
Показано, что задача решения нормальной системы (1) и задача минимизации функционала (2) {\it эквивалентны}. Таким образом через минимизацию (2) , будет найден вектор $\mathbf{x}$ решения (1).

\subsection{Алгоритм}
\begin{enumerate}
    \item \begin{enumerate}
        \item Задать начальное приближение $\mathbf{x}^{(0)}$ и допустимый уровень погрешности $\varepsilon>0$;
        \item Вычислить невязку начального приближения $\mathbf{r}^{(0)}=\mathbf{b-Ax}^{(0)}$;
        \item Положить $\mathbf{p}^{(0)}=\mathbf{r}^{(0)},~k=0$ (номер итерации);
    \end{enumerate}
    \item \begin{enumerate}
        \item Вычислить $\mathbf{q}^{(k)}=\mathbf{Ap}^{(k)}$;
        \item Вычислить шаговый скаляр $\alpha_k=\dfrac{(\mathbf{r}^{(k)}, \mathbf{p}^{(k)})}{(\mathbf{q}^{(k)}, \mathbf{p}^{(k)})}$;
        \item Вычислить очередное приближение $\mathbf{x}^{(k+1)}=\mathbf{x}^{(k)}+\alpha_k\mathbf{p}^{(k)}$;
        \item Вычислить невязку $(k+1)$-го приближения $\mathbf{r}^{(k+1)}=\mathbf{r}^{(k)}-\alpha_k\mathbf{q}^{(k)}$;
        \item Проверить выполнение $||\mathbf{r}^{(k+1)}||_2 \le \varepsilon ~~\implies ~ \blacksquare$
    \end{enumerate}
    \item \begin{enumerate}
        \item Вычислить скаляр $\beta_k=\dfrac{(\mathbf{r}^{(k+1)}, \mathbf{q}^{(k)})}{(\mathbf{q}^{(k)}, \mathbf{p}^{(k)})}$;
        \item Вычислить новый вектор градиента: $\mathbf{p}^{(k+1)}=\mathbf{r}^{(k+1)}-\beta_k\mathbf{p}^{(k)}$;
        \item Увеличить счётчик $k:=k+1$ и вернуться к 2.1.
    \end{enumerate}
\end{enumerate}

\clearpage
\section{Тестовый пример}
\subsection{Задание исходных данных}
Из положительно определённых квадратных матриц с небольшими целочисленными значениями ранга 4 методом перебора выберем $\mathbf{A}$ с достаточно малым числом обусловленности, зададим также произвольный $\mathbf{x}$ и вычислим $\mathbf{b}$:
\[
\mathbf{A} =
\begin{pmatrix}
    5 & -1 & -1 &  0\\
   -1 &  6 &  1 & -1\\
   -1 &  1 &  6 &  1\\
    0 & -1 &  1 &  7
\end{pmatrix}; ~~~~~
\begin{matrix}
    \texttt{cond(A)}=&1.812;\\
    \texttt{det(A)} =&1089;
\end{matrix} ~~~~~
\mathbf{x} = \begin{pmatrix} -2 \\ 0 \\ 2\\ 1 \end{pmatrix}; ~~~~~
\mathbf{b=Ax} = \begin{pmatrix} -12 \\ 3 \\ 15 \\ 9 \end{pmatrix};
\]

\subsection{Расчёт}
\begin{enumerate}
    \item \begin{enumerate}
        \item $$\mathbf{x}^{(0)} :=
        \begin{pmatrix} 1\\ 1\\ 1\\ 1 \end{pmatrix};$$

        \item $$\mathbf{r}^{(0)} :=
            \mathbf{b-Ax}^{(0)} = \begin{pmatrix} -15\\ -2\\ 8 \\ 2 \end{pmatrix}; ~~~
        ||\mathbf{r}^{(0)}||_2 = \mathbf{17.234};$$

        \item $$\mathbf{p}^{(0)} := \mathbf{r}^{(0)};$$
    \end{enumerate}

    \item \begin{enumerate}
        \item $$\mathbf{q}^{(0)} :=
            \mathbf{Ap}^{(0)} = \begin{pmatrix} -81\\ 9\\ 63\\ 24 \end{pmatrix};$$

        \item $$\alpha_0 :=
            \dfrac{(\mathbf{r}^{(0)}, \mathbf{p}^{(0)})}{(\mathbf{q}^{(0)}, \mathbf{p}^{(0)})} = 0.170;$$

        \item $$\mathbf{x}^{(1)} :=
            \mathbf{x}^{(0)}+\alpha_0\mathbf{p}^{(0)} = \begin{pmatrix} -1.550\\ 0.660\\ 2.360\\ 1.340 \end{pmatrix};$$

        \item $$\mathbf{r}^{(1)} :=
            \mathbf{r}^{(0)}-\alpha_0\mathbf{q}^{(0)} = \begin{pmatrix} -1.230\\ -3.530\\ -2.710 \\ -2.080 \end{pmatrix};$$

        \item $$||\mathbf{r}^{(1)}||_2 = \mathbf{5.064}$$
    \end{enumerate}

    \item \begin{enumerate}
        \item $$\beta_0 :=
            \dfrac{(\mathbf{r}^{(1)}, \mathbf{q}^{(0)})}{(\mathbf{q}^{(0)}, \mathbf{p}^{(0)})} = -0.087;$$

        \item $$\mathbf{p}^{(1)} :=
            \mathbf{r}^{(1)}-\beta_0\mathbf{p}^{(0)} = \begin{pmatrix} -2.535\\-3.704\\-2.014\\-1.906 \end{pmatrix};$$
    \end{enumerate}

    \item \begin{enumerate}
        \item $$\mathbf{q}^{(1)} :=
            \mathbf{Ap}^{(1)} = \begin{pmatrix} -6.957\\-19.797\\-15.159\\-11.652 \end{pmatrix};$$

        \item $$\alpha_1 :=
            \dfrac{(\mathbf{r}^{(1)}, \mathbf{p}^{(1)})}{(\mathbf{q}^{(1)}, \mathbf{p}^{(1)})} = 0.178;$$

        \item $$\mathbf{x}^{(2)} :=
            \mathbf{x}^{(1)}+\alpha_1\mathbf{p}^{(1)} = \begin{pmatrix} -2.001\\0.001\\2.002\\1.001 \end{pmatrix};$$

        \item $$\mathbf{r}^{(2)} :=
            \mathbf{r}^{(1)}-\alpha_1\mathbf{q}^{(1)} = \begin{pmatrix} 0.008\\-0.006\\-0.012\\-0.006 \end{pmatrix};$$

        \item $$||\mathbf{r}^{(2)}||_2 = \mathbf{0.017};$$
    \end{enumerate}

    \item \begin{enumerate}
        \item $$\beta_1 :=
            \dfrac{(\mathbf{r}^{(2)}, \mathbf{q}^{(1)})}{(\mathbf{q}^{(1)}, \mathbf{p}^{(1)})} = 0.002;$$

        \item $$\mathbf{p}^{(2)} :=
            \mathbf{r}^{(2)}-\beta_1\mathbf{p}^{(1)} = \begin{pmatrix} 0.013\\0.001\\-0.008\\-0.002 \end{pmatrix};$$
    \end{enumerate}

    \item \begin{enumerate}
        \item $$\mathbf{q}^{(2)} :=
            \mathbf{Ap}^{(2)} = \begin{pmatrix}0.072\\-0.013\\-0.062\\-0.023\end{pmatrix};$$

        \item $$\alpha_2 :=
            \dfrac{(\mathbf{r}^{(2)}, \mathbf{p}^{(2)})}{(\mathbf{q}^{(2)}, \mathbf{p}^{(2)})} = 0.141;$$

        \item $$\mathbf{x}^{(3)} :=
            \mathbf{x}^{(2)}+\alpha_2\mathbf{p}^{(2)} = \begin{pmatrix}-1.999\\0.001\\2.001\\1.001\end{pmatrix};$$

        \item $$\mathbf{r}^{(3)} :=
            \mathbf{r}^{(2)}-\alpha_2\mathbf{q}^{(2)} = \begin{pmatrix}-0.002\\-0.004\\-0.003\\-0.003 \end{pmatrix};$$

        \item $$||\mathbf{r}^{(3)}||_2 = \mathbf{0.006};$$
   \end{enumerate}

    \item \begin{enumerate}
        \item $$\beta_2 :=
            \dfrac{(\mathbf{r}^{(3)}, \mathbf{q}^{(2)})}{(\mathbf{q}^{(2)}, \mathbf{p}^{(2)})} = 0.111;$$

        \item $$\mathbf{p}^{(3)} :=
            \mathbf{r}^{(3)}-\beta_2\mathbf{p}^{(2)} = \begin{pmatrix} 0.003\\0.004\\-0.002\\-0.003 \end{pmatrix};$$
    \end{enumerate}
\end{enumerate}

\subsection{Комментарий}
По результатам 4х циклов в п. 6.5 расчёта получен модуль невязки $||\mathbf{r}^{(3)}||_2 = \mathbf{0.006}$, фактическая ошибка расчёта составила: $\varepsilon = ||\mathbf{x}^{(3)} - \mathbf{x}||_2 = \mathbf{0.0020}$

\clearpage
\section{Программная реализация на C++}
\subsection{Основная функция}
\begin{lstlisting}[language=c++]
pair<Vec, int> conjugateGradientMethod(const Mtr& A, const Vec& b, double eps, int maxIter){
    Vec x_k = Vec(b.size(), 1);
    Vec r = vecSum(1, b, -1, multiplyMatrixVector(A, x_k));
    Vec p = r;
    int k = 0;

    while (k <= maxIter){
        Vec q = multiplyMatrixVector(A, p);
        double pq_denom = dotProduct(p, q);
        double alpha = dotProduct(r, p) / pq_denom;
        x_k = vecSum(1, x_k, alpha, p);
        r = vecSum(1, r, -alpha, q);
        if (euclideanNorm(r) <= eps){
            return make_pair(x_k, k);
        }
        double beta = dotProduct(r, q) / pq_denom;
        p = vecSum(1, r, -beta, p);
        k++;
    }
    return make_pair(x_k, -1);
}
\end{lstlisting}
\subsection{Модульная структура программы}
Некоторое содержимое заголовочного файла, используемого в программной реализации, названия функций самообъясняющие:
\begin{lstlisting}[language=c++]
using namespace std;
typedef vector<double> Vec;
typedef vector<Vec> Mtr;

Vec generateRandomVector(int n, double lower=0, double upper=1);
Mtr generateRandomMatrix(int rows, int cols, double lower=0, double upper=1);
Mtr generateRndSymPos(int n, double cond);
Mtr transpose(const Mtr& matrix);

double dotProduct(const Vec& vector1, const Vec& vector2);
double euclideanNorm(const Vec& vector);

Vec vecSum(double c1, const Vec& vector1, double c2, const Vec& vector2); // c1*vector1 + c2*vector2
Vec multiplyMatrixVector(const Mtr& matrix, const Vec& vector);

pair<Vec, int> conjugateGradientMethod(const Mtr& A, const Vec& b, double eps, int maxIter);
\end{lstlisting}

\clearpage
\section{Анализ сходимости метода}
\subsection{Методика расчёта}
При анализе сходимости метода важно не забывать об условиях его применимости. МСГ применим к СЛАУ с положительно определёнными симметричными матрицами. Однако если подойти к вопросу формально и просто заполнить квадратную матрицу $\mathsf{A}$ ранга n случайными (для определённости, равномерно распределёнными между 0 и 1) числами и исследовать сходимость МСГ на матрице $\mathsf{A} \cdot \mathsf{A}^T$, то обусловленность и определитель такой системы будут непредсказуемы и метод не покажет сходимость за $\le n$ шагов. По этой причине анализе сходимости были проведён на специально {\it сконструированных} матрицах вида $\mathbf{A = H}^T\cdot \mathbf{D \cdot H}$, где $\mathbf{D}$ -- диагональная, a $\mathbf{H = E} - 2\mathbf{ww}^T$ -- унитарная матрица, полученная из преобразования Хаусхолдера над случайным нормированным вектором $\mathbf{w}$ нужной длины. Если $\mathcal{C}\ge1$ -- выбранное число обусловленности, то $\mathbf{D}$ принимает вид:
$$\mathbf{D} =
\begin{pmatrix}
    \mathcal{C} &   &   &      &       &\\
                &r_2&   &      &       &\\
                &   &r_3&      &       &\\
                &   &   &\ddots&       &\\
                &   &   &      &r_{n-1}&\\
                &   &   &      &       &1
\end{pmatrix},$$
в которой $\{r_2 \ldots r_{n-1}\}$ -- случайные вещественные числа в диапазоне $[1, \mathcal{C}]$, что позволяет задать матрицу с желаемым числом обусловленности.

\subsection{Графики}
Ниже приведены графики сходимости метода для матриц нескольких фиксированных размеров. Количество шагов усреднено по 10 повторам, но ошибка выбиралась по худшему значению.
\begin{figure}[H]
    \centering
    \caption{Сходимость МСГ на случайных симметричных п.о. матрицах $\mathsf{A} \cdot \mathsf{A}^T$}
    \includegraphics[width=1\linewidth]{pics/conv_rand}
    \label{fig:1}
\end{figure}
\begin{figure}[H]
    \centering
    \caption{Сходимость МСГ на {\it сконструированных} матрицах, \texttt{cond}=500}
    \includegraphics[width=1\linewidth]{pics/conv_cond=500}
    \label{fig:2}
\end{figure}
\begin{figure}[H]
    \centering
    \caption{Сходимость МСГ на {\it сконструированных} матрицах, \texttt{cond}=50}
    \includegraphics[width=1\linewidth]{pics/conv_cond=50}
    \label{fig:3}
\end{figure}
\begin{figure}[H]
    \centering
    \caption{Сходимость МСГ на {\it сконструированных} матрицах, \texttt{cond}=5}
    \includegraphics[width=1\linewidth]{pics/conv_cond=5}
    \label{fig:4}
\end{figure}

\subsection{Комментарий}
Приведённые графики иллюстрируют влияние числа обусловленности на сходимость метода сопряжённых градиентов. Критерием выхода из алгоритма МСГ является норма вектора невязки, модифицируемого на каждом шаге. Её значение в общем случае не совпадает с нормой фактического вектора невязки, которую можно рассчитать апостериори. Также она может отличаться от фактической ошибки полученного решения СЛАУ. Это различие заметно на графиках сходимости плохо обусловленных систем.

На хорошо обусловленных же системах предписанное теорией количество шагов до достижения желаемой точности выполняется на практике, и точность совпадает с фактической ошибкой вплоть до $10^{-13}\ldots10^{-14}$.


\end{document}