\section{Разложение Холецкого}

\subsection{Теория}

Факторизация Холецкого (метод квадратного корня) -- представление положительно определённой эрмитовой матрицы $A$ в виде $A = LL^T$, где $L$ -- нижняя треугольная матрица. Показано, что для подходящей матрицы такое разложение всегда существует и единственно, для вещественных матриц достаточным условием вещественного разложения является преобладание диагональных элементов над недиагональными.

\subsection{Алгоритм полного разложения}

Обозначая исходную матрицу: $a_{m,k}$, матрицу разложения: $l_{m,k}$, где $m = 1..n$ -- индексы строк, $k=1..n$ -- индексы столбцов, и начиная с верхнего левого элемента:
\begin{enumerate}
    \item Для первого элемента на главной диагонали: $l_{1,1} := \sqrt{a_{1,1}}$;

    \item Для элементов первого столбца: $l_{j,1} := \frac{a_{j,1}}{l_{1,1}}, ~~~ j=2..n$;

    \item Далее в цикле по столбцам $i=2..n$:
    \begin{enumerate}
        \item Расчёт диагонального элемента: $l_{i,i} := \sqrt{a_{i,i} - \sum_{p=1}^{i-1}l_{i,p}^2}$;

        \item Расчёт элементов под диагональным : $l_{j,i} := \frac{1}{l_{i,i}}(a_{j,i} - \sum_{p=1}^{i-1}l_{i,p}l_{j,p}), ~~~ i<n,~~j=i+1..n$;
    \end{enumerate}
\end{enumerate}

\subsection{Алгоритм неполного разложения (с порогом)}
Отличается от полного разложения добавлением проверки на шаге 3.2: если рассчитанный недиагональный элемент для новой матрицы $L$ не превосходит по абсолютной величине значение установленного порога $\theta$, то вместо такого элемента в матрицу $L$ на эту позицию записывают 0. Таким образом, если $\theta = 0$, то шаблон разложения копирует шаблон исходной матрицы (важно при использовании разреженных массивов), если же $\theta>0$, то в дополнение к этому из матрицы разложения устраняются элементы в наименьшей степени влияющие на характеристики матрицы. Установление ненулевого порога может быть оправдано при использовании разложения Холецкого в качестве предобуславливателя при решении СЛАУ в целях экономии вычислительных ресурсов.

\clearpage
\subsection{Тестовый пример}
Используем положительно определённую симметричную матрицу:
$$A = \begin{pmatrix}
    7.7  & 4.36 & 7.79  & 5.77 & 0.40 \\
    4.36 & 7.26 & 8.17  & 3.29 & 0    \\
    7.79 & 8.17 & 12.87 & 7.23 & 1.34 \\
    5.77 & 3.29 & 7.23  & 7.87 & 1.36 \\
    0.40 & 0    & 1.34  & 1.36 & 1.20
\end{pmatrix}$$

\begin{multicols}{3}
Полное разложение Холецкого:
\begin{enumerate}
    \item Квадратный корень из первого элемента:
    $$\begin{pmatrix}
        \sqrt{7.70} & 0 & 0 & 0 & 0 \\
        0           & 0 & 0 & 0 & 0 \\
        0           & 0 & 0 & 0 & 0 \\
        0           & 0 & 0 & 0 & 0 \\
        0           & 0 & 0 & 0 & 0
    \end{pmatrix}$$

    \item Расчёт первого столбца:
    $$\begin{pmatrix}
        2.77              & 0 & 0 & 0 & 0 \\
        \frac{4.36}{2.77} & 0 & 0 & 0 & 0 \\
        \frac{7.79}{2.77} & 0 & 0 & 0 & 0 \\
        \frac{5.77}{2.77} & 0 & 0 & 0 & 0 \\
        \frac{0.40}{2.77} & 0 & 0 & 0 & 0
    \end{pmatrix}$$

    \item Расчёт очередного диагонального элемента: $l_{2,2} = \sqrt{7.26 - 1.57^2} = 2.19$
    $$\begin{pmatrix}
        2.77 & 0    & 0 & 0 & 0 \\
        1.57 & 2.19 & 0 & 0 & 0 \\
        2.81 & 0    & 0 & 0 & 0 \\
        2.08 & 0    & 0 & 0 & 0 \\
        0.14 & 0    & 0 & 0 & 0
    \end{pmatrix}$$

    \item Расчёт очередного столбца под диагональю: $l_{j,2} := \frac{1}{2.19}(a_{j,2} - \sum_{p=1}^{i-1}l_{2,p}l_{j,p}),~~j=3..5$
    $$\begin{pmatrix}
        2.77&  0   & 0& 0& 0\\
        1.57&  2.19& 0& 0& 0\\
        2.81&  1.72& 0& 0& 0\\
        2.08&  0.01& 0& 0& 0\\
        0.14& -0.10& 0& 0& 0
    \end{pmatrix}$$
\end{enumerate}

\columnbreak
Неполное разложение Холецкого $\theta=0$:
\begin{enumerate}
    \item Квадратный корень из первого элемента:
    $$\begin{pmatrix}
        \sqrt{7.70} & 0 & 0 & 0 & 0 \\
        0           & 0 & 0 & 0 & 0 \\
        0           & 0 & 0 & 0 & 0 \\
        0           & 0 & 0 & 0 & 0 \\
        0           & 0 & 0 & 0 & 0
    \end{pmatrix}$$

    \item Расчёт первого столбца:
    $$\begin{pmatrix}
        2.77              & 0 & 0 & 0 & 0 \\
        \frac{4.36}{2.77} & 0 & 0 & 0 & 0 \\
        \frac{7.79}{2.77} & 0 & 0 & 0 & 0 \\
        \frac{5.77}{2.77} & 0 & 0 & 0 & 0 \\
        \frac{0.40}{2.77} & 0 & 0 & 0 & 0
    \end{pmatrix}$$

    \item Расчёт очередного диагонального элемента: $l_{2,2} = \sqrt{7.26 - 1.57^2} = 2.19$
    $$\begin{pmatrix}
        2.77 & 0    & 0 & 0 & 0 \\
        1.57 & 2.19 & 0 & 0 & 0 \\
        2.81 & 0    & 0 & 0 & 0 \\
        2.08 & 0    & 0 & 0 & 0 \\
        0.14 & 0    & 0 & 0 & 0
    \end{pmatrix}$$

    \item Расчёт очередного столбца под диагональю:$l_{j,2} := \frac{1}{2.19}(a_{j,2} - \sum_{p=1}^{i-1}l_{2,p}l_{j,p}),~~j=3..5$ кроме позиций нулей в $A$
    $$\begin{pmatrix}
        2.77 & 0          & 0 & 0 & 0 \\
        1.57 & 2.19       & 0 & 0 & 0 \\
        2.81 & 1.72       & 0 & 0 & 0 \\
        2.08 & 0.01       & 0 & 0 & 0 \\
        0.14 & \mathbf{0} & 0 & 0 & 0
    \end{pmatrix}$$
\end{enumerate}

\columnbreak
Неполное разложение Холецкого $\theta=0.2$:
\begin{enumerate}
    \item Квадратный корень из первого элемента:
    $$\begin{pmatrix}
        \sqrt{7.70} & 0 & 0 & 0 & 0 \\
        0           & 0 & 0 & 0 & 0 \\
        0           & 0 & 0 & 0 & 0 \\
        0           & 0 & 0 & 0 & 0 \\
        0           & 0 & 0 & 0 & 0
    \end{pmatrix}$$

    \item Расчёт первого столбца кроме позиций не проходящих порог $\theta$:
    $$\begin{pmatrix}
        2.77              & 0 & 0 & 0 & 0 \\
        \frac{4.36}{2.77} & 0 & 0 & 0 & 0 \\
        \frac{7.79}{2.77} & 0 & 0 & 0 & 0 \\
        \frac{5.77}{2.77} & 0 & 0 & 0 & 0 \\
        \mathbf{0}        & 0 & 0 & 0 & 0
    \end{pmatrix}$$

    \item Расчёт очередного диагонального элемента: $l_{2,2} = \sqrt{7.26 - 1.57^2} = 2.19$
    $$\begin{pmatrix}
        2.77       & 0    & 0 & 0 & 0 \\
        1.57       & 2.19 & 0 & 0 & 0 \\
        2.81       & 0    & 0 & 0 & 0 \\
        2.08       & 0    & 0 & 0 & 0 \\
        \mathbf{0} & 0    & 0 & 0 & 0
    \end{pmatrix}$$

    \item Расчёт очередного столбца под диагональю:$l_{j,2} := \frac{1}{2.19}(a_{j,2} - \sum_{p=1}^{i-1}l_{2,p}l_{j,p}),~~j=3..5$, кроме позиций не проходящих порог $\theta$:
    $$\begin{pmatrix}
        2.77       & 0          & 0 & 0 & 0 \\
        1.57       & 2.19       & 0 & 0 & 0 \\
        2.81       & 1.72       & 0 & 0 & 0 \\
        2.08       & \mathbf{0} & 0 & 0 & 0 \\
        \mathbf{0} & \mathbf{0} & 0 & 0 & 0
    \end{pmatrix}$$
\end{enumerate}
\end{multicols}

\clearpage

\begin{multicols}{3}
    \begin{enumerate}
    \setcounter{enumi}{4}
    \item Расчёт очередного диагонального элемента: $l_{3,3} = \sqrt{12.87 - (2.81^2+1.72^2)} = 1.43$
    $$\begin{pmatrix}
        2.77 & 0     & 0    & 0 & 0 \\
        1.57 & 2.19  & 0    & 0 & 0 \\
        2.81 & 1.72  & 0    & 0 & 0 \\
        2.08 & 0.01  & 0    & 0 & 0 \\
        0.14 & -0.10 & 0    & 0 & 0
    \end{pmatrix}$$
    \vspace{-0.5cm}
    \item Расчёт очередного столбца под диагональю: $l_{j,3} := \frac{1}{1.43}(a_{j,3} - \sum_{p=1}^{i-1}l_{3,p}l_{j,p}),~~j=4..5$
    $$\begin{pmatrix}
        2.77 & 0     & 0    & 0 & 0 \\
        1.57 & 2.19  & 0    & 0 & 0 \\
        2.81 & 1.72  & 1.43 & 0 & 0 \\
        2.08 & 0.01  & 0.96 & 0 & 0 \\
        0.14 & -0.10 & 0.78 & 0 & 0
    \end{pmatrix}$$
    \vspace{-0.5cm}
    \item Расчёт очередного диагонального элемента: $l_{4,4} = \sqrt{7.87 - (2.08^2+0.01^2+0.96^2)} = 1.62$
    $$\begin{pmatrix}
        2.77 & 0     & 0    & 0    & 0 \\
        1.57 & 2.19  & 0    & 0    & 0 \\
        2.81 & 1.72  & 1.43 & 0    & 0 \\
        2.08 & 0.01  & 0.96 & 1.62 & 0 \\
        0.14 & -0.10 & 0.78 & 0    & 0
    \end{pmatrix}$$
    \vspace{-0.5cm}
    \item Расчёт очередного столбца под диагональю: $l_{j,4} := \frac{1}{1.62}(a_{5,4} - \sum_{p=1}^{3}l_{3,p}l_{5,p})$
    $$\begin{pmatrix}
        2.77 & 0     & 0    & 0    & 0 \\
        1.57 & 2.19  & 0    & 0    & 0 \\
        2.81 & 1.72  & 1.43 & 0    & 0 \\
        2.08 & 0.01  & 0.96 & 1.62 & 0 \\
        0.14 & -0.10 & 0.78 & 0.19 & 0
    \end{pmatrix}$$
    \vspace{-0.5cm}
    \item Расчёт последнего диагонального элемента: $l_{5,5} = \sqrt{1.20 - (0.14^2+0.1^2+0.78^2+0.19^2)} = 0.72$
    $$\begin{pmatrix}
        2.77 & 0     & 0    & 0    & 0    \\
        1.57 & 2.19  & 0    & 0    & 0    \\
        2.81 & 1.72  & 1.43 & 0    & 0    \\
        2.08 & 0.01  & 0.96 & 1.62 & 0    \\
        0.14 & -0.10 & 0.78 & 0.19 & 0.72
    \end{pmatrix}$$
    \end{enumerate}


    \columnbreak
    \begin{enumerate}
    \setcounter{enumi}{4}
    \item Расчёт очередного диагонального элемента: $l_{3,3} = \sqrt{12.87 - (2.81^2+1.72^2)} = 1.43$
    $$\begin{pmatrix}
        2.77 & 0          & 0    & 0 & 0 \\
        1.57 & 2.19       & 0    & 0 & 0 \\
        2.81 & 1.72       & 1.43 & 0 & 0 \\
        2.08 & 0.01       & 0    & 0 & 0 \\
        0.14 & \mathbf{0} & 0    & 0 & 0
    \end{pmatrix}$$
    \vspace{-0.5cm}
    \item Расчёт очередного столбца под диагональю кроме позиций нулей в $A$:
    $$\begin{pmatrix}
        2.77 & 0          & 0    & 0 & 0 \\
        1.57 & 2.19       & 0    & 0 & 0 \\
        2.81 & 1.72       & 1.43 & 0 & 0 \\
        2.08 & 0.01       & 0.96 & 0 & 0 \\
        0.14 & \mathbf{0} & 0.65 & 0 & 0
    \end{pmatrix}$$
    \vspace{-0.5cm}
    \item Расчёт очередного диагонального элемента: $l_{4,4} = \sqrt{7.87 - (2.087^2+0.01^2+0.96^2)} = 1.62$
    $$\begin{pmatrix}
        2.77 & 0          & 0    & 0    & 0 \\
        1.57 & 2.19       & 0    & 0    & 0 \\
        2.81 & 1.72       & 1.43 & 0    & 0 \\
        2.08 & 0.01       & 0.96 & 1.62 & 0 \\
        0.14 & \mathbf{0} & 0.65 & 0    & 0
    \end{pmatrix}$$
    \vspace{-0.5cm}
    \item Расчёт очередного столбца под диагональю кроме позиций нулей в $A$:
    $$\begin{pmatrix}
        2.77 & 0          & 0    & 0    & 0 \\
        1.57 & 2.19       & 0    & 0    & 0 \\
        2.81 & 1.72       & 1.43 & 0    & 0 \\
        2.08 & 0.01       & 0.96 & 1.62 & 0 \\
        0.14 & \mathbf{0} & 0.65 & 0.27 & 0
    \end{pmatrix}$$
    \vspace{-0.5cm}
    \item Расчёт последнего диагонального элемента: $l_{5,5} = \sqrt{1.20 - (0.14^2+0.65^2+0.27^2)} = 0.82$
    $$\begin{pmatrix}
        2.77 & 0          & 0    & 0    & 0    \\
        1.57 & 2.19       & 0    & 0    & 0    \\
        2.81 & 1.72       & 1.43 & 0    & 0    \\
        2.08 & 0.01       & 0.96 & 1.62 & 0    \\
        0.14 & \mathbf{0} & 0.65 & 0.27 & 0.82
    \end{pmatrix}$$
    \end{enumerate}


    \columnbreak
    \begin{enumerate}
    \setcounter{enumi}{4}
        \item Расчёт очередного диагонального элемента: $l_{3,3} = \sqrt{12.87 - (2.81^2+1.72^2)} = 1.43$
    $$\begin{pmatrix}
        2.77       & 0          & 0    & 0 & 0 \\
        1.57       & 2.19       & 0    & 0 & 0 \\
        2.81       & 1.72       & 1.43 & 0 & 0 \\
        2.08       & \mathbf{0} & 0    & 0 & 0 \\
        \mathbf{0} & \mathbf{0} & 0    & 0 & 0
    \end{pmatrix}$$
    \vspace{-0.5cm}
    \item Расчёт очередного столбца под диагональю, кроме позиций не проходящих порог $\theta$:
    $$\begin{pmatrix}
        2.77       & 0          & 0    & 0 & 0 \\
        1.57       & 2.19       & 0    & 0 & 0 \\
        2.81       & 1.72       & 1.43 & 0 & 0 \\
        2.08       & \mathbf{0} & 0.98 & 0 & 0 \\
        \mathbf{0} & \mathbf{0} & 0.94 & 0 & 0
    \end{pmatrix}$$
    \vspace{-0.5cm}
    \item Расчёт очередного диагонального элемента: $l_{4,4} = \sqrt{7.87 - (2.08^2+0.98^2)} = 1.61$
    $$\begin{pmatrix}
        2.77       & 0          & 0    & 0    & 0 \\
        1.57       & 2.19       & 0    & 0    & 0 \\
        2.81       & 1.72       & 1.43 & 0    & 0 \\
        2.08       & \mathbf{0} & 0.98 & 1.61 & 0 \\
        \mathbf{0} & \mathbf{0} & 0.94 & 0    & 0
    \end{pmatrix}$$
    \vspace{-0.5cm}
    \item Расчёт очередного столбца под диагональю кроме позиций не проходящих порог $\theta$:
    $$\begin{pmatrix}
        2.77       & 0          & 0    & 0    & 0 \\
        1.57       & 2.19       & 0    & 0    & 0 \\
        2.81       & 1.72       & 1.43 & 0    & 0 \\
        2.08       & \mathbf{0} & 0.98 & 1.61 & 0 \\
        \mathbf{0} & \mathbf{0} & 0.94 & 0.28 & 0
    \end{pmatrix}$$
    \vspace{-0.5cm}
    \item Расчёт последнего диагонального элемента: $l_{5,5} = \sqrt{1.20 - (0.94^2 + 0.28^2)} = 0.49$
    $$\begin{pmatrix}
        2.77       & 0          & 0    & 0    & 0    \\
        1.57       & 2.19       & 0    & 0    & 0    \\
        2.81       & 1.72       & 1.43 & 0    & 0    \\
        2.08       & \mathbf{0} & 0.98 & 1.61 & 0    \\
        \mathbf{0} & \mathbf{0} & 0.94 & 0.28 & 0.49
    \end{pmatrix}$$
    \end{enumerate}
\end{multicols}
Оценка максимального уклонения обратного разложения по формуле $\max(LL^T - A)$:
\begin{enumerate}
    \item Полное разложение, $\max(|err|) = 0.03$:

    \item Разложение с нулевым порогом $\theta=0$, $\max(|err|) = 0.22$:

    \item Разложение с ненулевым порогом $\theta=0.2$, $\max(|err|) = 0.4$:
\end{enumerate}
