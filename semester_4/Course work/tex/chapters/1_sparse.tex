\section{Разреженные матрицы}

\subsection{Теория}

При решении некоторых классов задач возникает необходимость работать с массивами, заполненными преимущественно нулями. Отношение числа ненулевых элементов к общему их количеству называют {\it плотностью} ({\it density}), отношение нулевых -- к общему количеству -- {\it разреженностью} ({\it sparsity}). Не существует единого мнения по поводу того, какая плотность/разреженность является пороговой, чтобы считать массив разреженным, обычно это решается для каждой задачи индивидуально. Для сокращения потребления ресурсов вычислительной техники используют особое представление таких массивов в памяти. Двумерные разреженные массивы называются разреженными матрицами.

Для хранения разреженных матриц используются несколько техник:
\begin{itemize}
    \item Список списков (LIL — List of Lists):

    \item Список координат (COO — Coordinate list);

    \item Сжатое хранение строкой (CSR — Compressed Sparse Row, CRS — Compressed Row Storage, Йельский формат);

    \item Сжатое хранение столбцом (CSС — Compressed Sparse Column, CСS — Compressed Column Storage);

    \item Особые приёмы учитывающие структуру заполнения матрицы: блочную, одно- и мульти-диагональную и т.д.;

    \item Словарь по ключам (DOK — Dictionary of Keys).
\end{itemize}

Последняя из которых используется в данной работе.