\section{Метод сопряжённых градиентов}

\subsection{Теория}

Существует класс методов итерационного решения СЛАУ, называемый {\bf методами вариационного типа}. Их обоснование производится с привлечением теории оптимизации, т.к. решение системы сводится к решению эквивалентной экстремальной задачи. Пусть:
\begin{equation}
    \mathbf{Ax=b}
\end{equation}
нормальная n-мерная система, т.е. $\mathbf{A}$ -- положительно определённая симметричная матрица и пусть $(\cdot, \cdot)$ -- скалярное произведение в пространстве $\mathbb{R}^n$. Образуем квадратичный функционал:
\begin{equation}
    \mathbf{\Phi}(\mathbf{x})=(\mathbf{Ax}, \mathbf{x}) - 2(\mathbf{b}, \mathbf{x}) + c,
    ~~\text{ где }~~ c \in \mathbb{R}^1 ~~\text{ (произвольная постоянная)}.
\end{equation}
Показано, что задача решения нормальной системы (1) и задача минимизации функционала (2) {\it эквивалентны}. Таким образом через минимизацию (2) , будет найден вектор $\mathbf{x}$ решения (1).

\subsubsection{Алгоритм}
\begin{enumerate}
    \item \begin{enumerate}
        \item Задать начальное приближение $\mathbf{x}^{(0)}$ и допустимый уровень погрешности $\varepsilon>0$;
        \item Вычислить невязку начального приближения $\mathbf{r}^{(0)}=\mathbf{b-Ax}^{(0)}$;
        \item Положить $\mathbf{p}^{(0)}=\mathbf{r}^{(0)},~k=0$ (номер итерации);
    \end{enumerate}
    \item \begin{enumerate}
        \item Вычислить $\mathbf{q}^{(k)}=\mathbf{Ap}^{(k)}$;
        \item Вычислить шаговый скаляр $\alpha_k=\dfrac{(\mathbf{r}^{(k)}, \mathbf{p}^{(k)})}{(\mathbf{q}^{(k)}, \mathbf{p}^{(k)})}$;
        \item Вычислить очередное приближение $\mathbf{x}^{(k+1)}=\mathbf{x}^{(k)}+\alpha_k\mathbf{p}^{(k)}$;
        \item Вычислить невязку $(k+1)$-го приближения $\mathbf{r}^{(k+1)}=\mathbf{r}^{(k)}-\alpha_k\mathbf{q}^{(k)}$;
        \item Проверить выполнение $||\mathbf{r}^{(k+1)}||_2 \le \varepsilon ~~\implies ~$ вернуть $\mathbf{x}^{(k+1)} \blacksquare$
    \end{enumerate}
    \item \begin{enumerate}
        \item Вычислить скаляр $\beta_k=\dfrac{(\mathbf{r}^{(k+1)}, \mathbf{q}^{(k)})}{(\mathbf{q}^{(k)}, \mathbf{p}^{(k)})}$;
        \item Вычислить новый вектор градиента: $\mathbf{p}^{(k+1)}=\mathbf{r}^{(k+1)}-\beta_k\mathbf{p}^{(k)}$;
        \item Увеличить счётчик $k:=k+1$ и вернуться к 2.1.
    \end{enumerate}
\end{enumerate}

\subsection{Предобусловленный метод сопряжённых градиентов}
Несмотря на то, что метод сопряжённых градиентов можно отнести к прямым, т.к. показано, что при любом начальном векторе $\mathbf{x}^{(0)}$ n-мерной линейной системы минимум функционала (2) достигается ровно за n шагов, метод используется как итерационный (это видно и из приведённого алгоритма 3.1.1), т.к. во-первых, реальный вычислительный процесс может быть довольно далек от идеального и, вследствие неизбежных ошибок округления, на n-м шаге может быть не достигнута нужная точность. Во-вторых, если размерность n решаемой задачи велика, число шагов, достаточное для получения решения системы с нужной точностью (т.е. выход по критерию на шаге 2.5), может оказаться значительно меньше n.

Вообще, метод сопряженных градиентов работает хорошо для матриц, которые либо хорошо обусловлены, либо имеют лишь небольшое число различных собственных значений. Чтобы добиться этого, применим метод сопряжённых градиентов к преобразованной системе:

\begin{equation}
    \mathbf{\tilde{A}z=\tilde{b}}
\end{equation}
где $\mathbf{\tilde{A} = C^{-1}AC^{-1}, ~~z = Cx,~~ \tilde{b} = C^{-1}b}$, a $\mathbf{C}$ -- симметричная положительно определённая матрица.

В качестве предобуславливателя можно использовать матрицу, полученную из неполного разложения Холецкого матрицы $\mathbf{A}$.

\subsubsection{Алгоритм предобусловленного МСГ на неполном преобразовании Холецкого}
\begin{enumerate}
    \item Вычислить неполное преобразование Холецкого $\mathbf{C_\theta}$ для матрицы $\mathbf{A}$ по алгоритму 2.3;

    \item \begin{enumerate}
        \item Задать начальное приближение $\mathbf{x}^{(0)}$ и допустимый уровень погрешности $\varepsilon>0$;
        \item Вычислить невязку начального приближения $\mathbf{r}^{(0)}=\mathbf{b-Ax}^{(0)}$;

        \item Вычислить преобразованный вектор неизвестных $\mathbf{z}^{(0)}$:
        \begin{enumerate}
            \item Прямым ходом метода Гаусса решить систему $\mathbf{C_\theta y^{(0)} = r}^{(0)}$ относительно $\mathbf{y}^{(0)}$;

            \item Обратным ходом метода Гаусса решить систему $\mathbf{C_\theta^T z^{(0)} = y}^{(0)}$ относительно $\mathbf{z}^{(0)}$;
        \end{enumerate}

        \item Положить $\mathbf{p}^{(0)}=\mathbf{z}^{(0)},~k=0$ (номер итерации);
    \end{enumerate}
    \item \begin{enumerate}
        \item Вычислить $\mathbf{q}^{(k)}=\mathbf{Ap}^{(k)}$;
        \item Вычислить шаговый скаляр $\alpha_k=\dfrac{(\mathbf{z}^{(k)}, \mathbf{r}^{(k)})}{(\mathbf{q}^{(k)}, \mathbf{p}^{(k)})}$;
        \item Вычислить очередное приближение $\mathbf{x}^{(k+1)}=\mathbf{x}^{(k)}+\alpha_k\mathbf{p}^{(k)}$;
        \item Вычислить невязку $(k+1)$-го приближения $\mathbf{r}^{(k+1)}=\mathbf{r}^{(k)}-\alpha_k\mathbf{q}^{(k)}$;
        \item Проверить выполнение $||\mathbf{r}^{(k+1)}||_2 \le \varepsilon ~~\implies ~$ вернуть $\mathbf{x}^{(k+1)} \blacksquare$
    \end{enumerate}
    \item \begin{enumerate}
        \item Вычислить новый преобразованный вектор неизвестных $\mathbf{z}^{(k+1)}$:
        \begin{enumerate}
            \item Прямым ходом метода Гаусса решить систему $\mathbf{C_\theta y^{(k+1)} = r^{(k+1)}}$ относительно $\mathbf{y}^{(k+1)}$;

            \item Обратным ходом метода Гаусса решить систему $\mathbf{C_\theta^T z^{(k+1)} = y}^{(k+1)}$ относительно $\mathbf{z}^{(k+1)}$;
        \end{enumerate}
        \item Вычислить скаляр $\beta_k=\dfrac{(\mathbf{r}^{(k+1)}, \mathbf{z}^{(k+1)})}{(\mathbf{z}^{(k)}, \mathbf{r}^{(k)})}$;
        \item Вычислить новый вектор градиента: $\mathbf{p}^{(k+1)}=\mathbf{z}^{(k+1)}+\beta_k\mathbf{p}^{(k)}$;
        \item Увеличить счётчик $k:=k+1$ и вернуться к 3.1.
    \end{enumerate}
\end{enumerate}


\clearpage

\subsection{Тестовые примеры}
\subsubsection{Без предобуславливателя}
Из положительно определённых квадратных матриц с небольшими целочисленными значениями ранга 4 методом перебора выберем $\mathbf{A}$ с достаточно малым числом обусловленности, зададим также произвольный $\mathbf{x}$ и вычислим $\mathbf{b}$:
\begin{equation}
    \mathbf{A} =
    \begin{pmatrix}
        5  & -1 & -1 & 0  \\
        -1 & 6  & 1  & -1 \\
        -1 & 1  & 6  & 1  \\
        0  & -1 & 1  & 7
    \end{pmatrix}; ~~~~~
    \begin{matrix}
    \texttt{cond(A)}=&1.812;\\
    \texttt{det(A)} =&1089;
    \end{matrix} ~~~~~
    \mathbf{x} = \begin{pmatrix} -2 \\ 0 \\ 2\\ 1 \end{pmatrix}; ~~~~~
    \mathbf{b=Ax} = \begin{pmatrix} -12 \\ 3 \\ 15 \\ 9 \end{pmatrix};
\end{equation}

\begin{enumerate}
    \item \begin{enumerate}
        \item $$\mathbf{x}^{(0)} :=
        \begin{pmatrix} 1\\ 1\\ 1\\ 1 \end{pmatrix};$$

        \item $$\mathbf{r}^{(0)} :=
        \mathbf{b-Ax}^{(0)} = \begin{pmatrix} -15\\ -2\\ 8 \\ 2 \end{pmatrix}; ~~~
        ||\mathbf{r}^{(0)}||_2 = \mathbf{17.234};$$

        \item $$\mathbf{p}^{(0)} := \mathbf{r}^{(0)};$$
    \end{enumerate}

    \item \begin{enumerate}
        \item $$\mathbf{q}^{(0)} :=
        \mathbf{Ap}^{(0)} = \begin{pmatrix} -81\\ 9\\ 63\\ 24 \end{pmatrix};$$

        \item $$\alpha_0 :=
        \dfrac{(\mathbf{r}^{(0)}, \mathbf{p}^{(0)})}{(\mathbf{q}^{(0)}, \mathbf{p}^{(0)})} = 0.170;$$

        \item $$\mathbf{x}^{(1)} :=
        \mathbf{x}^{(0)}+\alpha_0\mathbf{p}^{(0)} = \begin{pmatrix} -1.550\\ 0.660\\ 2.360\\ 1.340 \end{pmatrix};$$

        \item $$\mathbf{r}^{(1)} :=
        \mathbf{r}^{(0)}-\alpha_0\mathbf{q}^{(0)} = \begin{pmatrix} -1.230\\ -3.530\\ -2.710 \\ -2.080 \end{pmatrix};$$

        \item $$||\mathbf{r}^{(1)}||_2 = \mathbf{5.064}$$
    \end{enumerate}

    \item \begin{enumerate}
        \item $$\beta_0 :=
        \dfrac{(\mathbf{r}^{(1)}, \mathbf{q}^{(0)})}{(\mathbf{q}^{(0)}, \mathbf{p}^{(0)})} = -0.087;$$

        \item $$\mathbf{p}^{(1)} :=
        \mathbf{r}^{(1)}-\beta_0\mathbf{p}^{(0)} = \begin{pmatrix} -2.535\\-3.704\\-2.014\\-1.906 \end{pmatrix};$$
    \end{enumerate}

    \item \begin{enumerate}
        \item $$\mathbf{q}^{(1)} :=
        \mathbf{Ap}^{(1)} = \begin{pmatrix} -6.957\\-19.797\\-15.159\\-11.652 \end{pmatrix};$$

        \item $$\alpha_1 :=
        \dfrac{(\mathbf{r}^{(1)}, \mathbf{p}^{(1)})}{(\mathbf{q}^{(1)}, \mathbf{p}^{(1)})} = 0.178;$$

        \item $$\mathbf{x}^{(2)} :=
        \mathbf{x}^{(1)}+\alpha_1\mathbf{p}^{(1)} = \begin{pmatrix} -2.001\\0.001\\2.002\\1.001 \end{pmatrix};$$

        \item $$\mathbf{r}^{(2)} :=
        \mathbf{r}^{(1)}-\alpha_1\mathbf{q}^{(1)} = \begin{pmatrix} 0.008\\-0.006\\-0.012\\-0.006 \end{pmatrix};$$

        \item $$||\mathbf{r}^{(2)}||_2 = \mathbf{0.017};$$
    \end{enumerate}
\end{enumerate}

\subsubsection{С предобуславливателем}

\begin{enumerate}
    \item Применим к матрице $\mathbf{A}$ разложение Холецкого с нулевым порогом:
    \begin{equation}
        \mathbf{C}_{\theta=0} = \begin{pmatrix}
            2.236  & 0      & 0     & 0     \\
            -0.447 & 2.408  & 0     & 0     \\
            -0.447 & 0.332  & 2.385 & 0     \\
            0      & -0.415 & 0.477 & 2.569
        \end{pmatrix}
    \end{equation}

    \item $$\mathbf{x}^{(0)} :=
    \begin{pmatrix} 1\\ 1\\ 1\\ 1 \end{pmatrix};$$

    \item $$\mathbf{r}^{(0)} :=
    \mathbf{b-Ax}^{(0)} = \begin{pmatrix} -15\\ -2\\ 8 \\ 2 \end{pmatrix}; ~~~
    ||\mathbf{r}^{(0)}||_2 = \mathbf{17.234};$$

    \item Два хода алгоритма Гаусса:
    \begin{enumerate}
        \item $$ \mathbf{Cy}^{(0)} = \mathbf{r}^{(0)}
        \begin{pmatrix}
            2.236  & 0      & 0     & 0     \\
            -0.447 & 2.408  & 0     & 0     \\
            -0.447 & 0.332  & 2.385 & 0     \\
            0      & -0.415 & 0.477 & 2.569
        \end{pmatrix}
        \begin{pmatrix} y^{(0)}_1\\ y^{(0)}_2\\ y^{(0)}_3 \\ y^{(0)}_4 \end{pmatrix} =
        \begin{pmatrix} -15\\ -2\\ 8 \\ 2 \end{pmatrix}~~\implies ~~
        \mathbf{y^{(0)}} = \begin{pmatrix}-6.708\\-2.076\\2.386\\0\end{pmatrix}$$
        \item $$\mathbf{C^Tz}^{(0)} = \mathbf{y}^{(0)}
        \begin{pmatrix}
            2.236 & -0.447 & -0.447 & 0      \\
            0     & 2.408  & 0.332  & -0.415 \\
            0     & 0      & 2.385  & 0.477  \\
            0     & 0      & 0      & 2.569
        \end{pmatrix}
        \begin{pmatrix} z^{(0)}_1\\z^{(0)}_2\\ z^{(0)}_3 \\ z^{(0)}_4 \end{pmatrix} =
        \begin{pmatrix}-6.708\\-2.076\\2.386\\0\end{pmatrix}~~\implies ~~
        \mathbf{z^{(0)}} = \begin{pmatrix}-3\\-1\\1\\0\end{pmatrix}$$
    \end{enumerate}

    \item $$\mathbf{p}^{(0)} := \mathbf{z}^{(0)};$$

    \item \begin{enumerate}
        \item $$\mathbf{q}^{(0)} :=
        \mathbf{Ap}^{(0)} = \begin{pmatrix} -15\\ -2\\ 8\\ 2 \end{pmatrix};$$

        \item $$\alpha_0 :=
        \dfrac{(\mathbf{z}^{(0)}, \mathbf{r}^{(0)})}{(\mathbf{q}^{(0)}, \mathbf{p}^{(0)})} = 1;$$

        \item $$\mathbf{x}^{(1)} :=
        \mathbf{x}^{(0)}+\alpha_0\mathbf{p}^{(0)} = \begin{pmatrix} -2\\ 0\\ 2\\ 1 \end{pmatrix};$$

         \item $$\mathbf{r}^{(1)} :=
        \mathbf{r}^{(0)}-\alpha_0\mathbf{q}^{(0)} = \begin{pmatrix} 0\\ 0\\ 0 \\ 0 \end{pmatrix};$$

        \item $$||\mathbf{r}^{(1)}||_2 = \mathbf{0} ~~\implies ~~ \text{метод сошёлся} ~~\implies ~~ \blacksquare$$
    \end{enumerate}
\end{enumerate}

\paragraph{Комментарий:} Применительно к данной матрице $\mathbf{A}$ неполное разложение Холецкого с нулевым порогом повело себя так же, как полное разложение Холецкого. Единственный нулевой элемент попал бы и в него. Так, метод сошёлся за единственный шаг, если не принимать во внимания дополнительные шаги, необходимые для двукратного применения метода Гаусса при расчёте $\mathbf{z}$ и шагов самого алгоритма факторизации. На практике, применять полное разложение не рационально, потому что в таком случае дополнительные вычислительные затраты на разложение и решение треугольных систем нивелируют выигрыш от одношагового схождения предобусловленного МСГ. Далее рассмотрим графики числа итераций в зависимости от величины порога в неполном разложении Холецкого.