\section{Решение стационарной задачи}

\subsection{Теория}

В приложениях часто возникает потребность решать уравнения вида:

\begin{equation}
    k\nabla^2 u + f(x,y,z,t) = \frac{\partial u}{\partial t},
\end{equation}

Двумерная стационарная версия которого при может при $k=1$ быть записана как
\begin{equation}
    \nabla^2 u + f(x,y) = 0,
\end{equation}
что может интерпретироваться как состояние плоской системы после окончания переходных процессов.

\subsubsection{Вариант 5}

\begin{equation}
    u = \sin(\pi x)\cos(\pi y)
\end{equation}

\begin{equation}
    -\nabla^2 u = f(x,y) = 2\pi^2\sin(\pi x)\cos(\pi y)
\end{equation}

Для заданной области $D = [0,1]\times[0,1]$ получим граничные условия:
\begin{itemize}
    \item $u(0,y) = 0$;
    \item $u(1,y) = 0$;
    \item $u(x,0) = -\sin(\pi x)$;
    \item $u(x,1) = \sin(\pi x)$;
\end{itemize}

Для проведения вычислительного эксперимента воспользуемся конечно-разностным методом. Для произвольного узла сетки с индексами i j аппроксимация частных производных функции конечно-разностным методом доставляет нам уравнение вида:
    \begin{equation}
    \frac{u_{i-1,j}-2u_{i,j}+u_{i+1,j}}{h^2}+\frac{u_{i,j-1}-2u_{i,j}+u_{i,j+1}}{h^2}+f(x_i, y_i) = 0
\end{equation}

Вектор решений представляет собой развернутую сетку. Развертка достигается за счет перенумерования точек сетки, каждой паре индексов $(i, j)$ ставится в соответствие индекс $k$. Ввиду конечности количества точек такая перенумерация осуществляется однозначно. Так, при $i = \{1..x_n\}, j = \{1..y_n\}, k = (j - 1) x_n + i$.

Полученную систему будем считать методом сопряжённых градиентов с предобуславливателем $\mathbf{C_\theta C_\theta}^T$, полученным из неполного разложения Холецкого по алгоритму 2.3.

\clearpage

\section{Графики}

Выбрав количество точек разбиения и приводя полученную систему к каноническому виду получим матрицу системы в виде блочной пятидиагональной матрицы.

\begin{figure}[H]
    \centering
    \includegraphics[width=0.7\linewidth]{pics/pp_full}
    \caption{Матрица модельной задачи 6*6 блоков}
\end{figure}
\begin{figure}[H]
    \centering
    \includegraphics[width=0.7\linewidth]{pics/pp_chol}
    \caption{Полное разложение Холецкого для матрицы модельной задачи.}
\end{figure}
\begin{figure}[H]
    \centering
    \includegraphics[width=0.7\linewidth]{pics/pp_1e-2}
    \caption{Частичное разложение Холецкого $\theta=0.01$.}
\end{figure}
\begin{figure}[H]
    \centering
    \includegraphics[width=0.7\linewidth]{pics/pp_4e-2}
    \caption{Частичное разложение Холецкого $\theta=0.04$.}
\end{figure}
\begin{figure}[H]
    \centering
    \includegraphics[width=0.7\linewidth]{pics/pp_16e-2}
    \caption{Частичное разложение Холецкого $\theta=0.16$.}
\end{figure}
\begin{figure}[H]
    \centering
    \includegraphics[width=0.7\linewidth]{pics/pp_ge0,5}
    \caption{Частичное разложение Холецкого $\theta>0.5$.}
\end{figure}

\clearpage
Экспериментируя с различными уровнями порогов для модельной задачи $15^2\times15^2$ (cond=103, density=0.021) и для случайной системы с симметричной матрицей, имеющей близкую величину плотности и число обусловленности были получены следующие зависимости:
\begin{figure}[H]
    \centering
    \includegraphics[width=1\linewidth]{pics/2ple}
    \caption{Зависимости количества шагов м.с.г для модельной задачи и для случайной матрицы со схожими характеристиками}
\end{figure}

\paragraph{Комментарий:} Как видим, при использовании предобуславливателя с уменьшением величины порога количество шагов м.с.г. стремится к 1 для обеих систем. Полное разложение гарантирует сходимость метода за один шаг (см. комментарий на стр.12). С увеличением же порога предобуславливатель работает всё хуже. При некоторой величине порога предобуславливатель начинает представлять из себя диагональную матрицу $\mathbf{D} = \mathbf{C}_{\theta_{lim}} \mathbf{C}^T_{\theta_{lim}}$ и далее повышать порог бессмысленно.

Однако можно видеть, что для случайной задачи даже диагональный предобуславливатель значительно улучшает сходимость м.с.г., чего нельзя сказать про матрицу стационарной задачи. При $\theta\ge 0.5$ её предобуславливатель принимает вид $2\mathbf{I}\cdot2\mathbf{I}^T$, что перестаёт влиять на сходимость использующего его м.с.г.

\clearpage
Для системы $30^2\times30^2$ было получено следующее решение:
\begin{figure}[H]
    \centering
    \includegraphics[width=0.8\linewidth]{pics/sol_3d}
    \caption{Решение уравнения Пуассона в заданной области (кроме граничных точек).}
\end{figure}
\begin{figure}[H]
    \centering
    \includegraphics[width=0.8\linewidth]{pics/err_3d}
    \caption{Уклонение решения от точного значения функции на заданной области ($err_{\max}=0.00029$)}
\end{figure}