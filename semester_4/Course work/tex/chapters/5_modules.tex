\section{Модульная структура программы}
Некоторые часто использованные функции:
\begin{itemize}

\item Просмотр структуры разреженной матрицы:

\lstinline|void spy(const spMtr& sA)|\\

\item Транспонирование:

\lstinline|spMtr T(const spMtr& M)|\\

\item Создание случайной симметричной разреженной матрицы с заданным числом обусловленности (n - сторона, cond - число обусловленности, density = $\dfrac{\text{ненулевые элементы}}{n^2}$):

\lstinline|spMtr generateRndSymPos(int n, double cond, double density)|\\

\item Полное разложение Холецкого (возвращает нижнетреугольную матрицу):

\lstinline|spMtr chol(const spMtr& M)|\\

\item Неполное разложение Холецкого с порогом (возвращает нижнетреугольную матрицу):

\lstinline|spMtr ichol(const spMtr& M, double threshold)|\\

\item Прямой и обратный прогон метода Гаусса для нижнетреугольной матрицы (предобуславлевателя):

\lstinline|Vec solve_L_U(const spMtr& L, const Vec& b)|\\

\item Метод сопряжённых градиентов с заданным максимальным количеством шагов (A -- матрица системы, b -- вектор правой части, eps -- желаемая точность решения системы, maxIter -- максимальное число итераций до завершения метода, вернёт пару (решение, количество шагов)):

\lstinline|pair<Vec, int> cg(const spMtr& A, const Vec& b, double eps, int maxIter)|\\

\item Предобусловденный метод сопряжённых градиентов с заданным максимальным количеством шагов (то же, что и в cg, L -- нижнетреугольная матрица разложения Холецкого):

\lstinline|pair<Vec, int> pcg(const spMtr& A, const spMtr& L, const Vec& b, double eps, int maxIter)|\\

\item Генератор матрицы для СЛАУ стационарной задачи (вернет блочно диагональную матрицу для решения 2-мерного стационарного уравнения Пуассона $n^2\times n^2$):

\lstinline|spMtr block5diag(size_t n)|\\

\item Расчёт вектора длины $n^2$ правой части с выбранным шагом (вернёт вектор правой части, посчитанный на основе заданных граничных условий и взятия частных производных методом конечных разностей):

\lstinline|Vec estimate_b(size_t n)|

\item Так же для удобства написания алгоритмов с матрицами и векторами для них были перегружены все необходимые алгебраические операторы: \lstinline|+  -  *  +=  -=|
\end{itemize}

