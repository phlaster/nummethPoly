\section{Модульная структура программы}
Некоторые часто использованные функции:

Просмотр структуры разреженной матрицы:

\lstinline|void spy(const spMtr& sA)|

Транспонирование:

\lstinline|spMtr T(const spMtr& M)|

Создание случайной симметричной разреженной матрицы с заданным числом обусловленности:

\lstinline|spMtr generateRndSymPos(int n, double cond, double sparsity)|

Полное разложение Холецкого:

\lstinline|spMtr chol(const spMtr& M)|

Неполное разложение Холецкого с порогом:

\lstinline|spMtr chol(const spMtr& M, double threshold)|

Прямой и обратный прогон метода гаусса для нижнетреугольной матрицы (предобуславлевателя):

\lstinline|Vec solve_L_U(const spMtr& L, const Vec& b)|

Метод сопряжённых градиентов с заданным максимальным количеством шагов:

\lstinline|pair<Vec, int> cg(const spMtr& A, const Vec& b, double eps, int maxIter)|

Предобусловденный метод сопряжённых градиентов с заданным максимальным количеством шагов:

\lstinline|pair<Vec, int> pcg(const spMtr& A, const spMtr& L, const Vec& b, double eps, int maxIter)|

Генератор матрицы для СЛАУ стационарной задачи:

\lstinline|spMtr block5diag(size_t m, size_t n)|

Расчёт вектора правой части с выбранным шагом:

\lstinline|Vec estimate_b(size_t n)|
