\section{Теория}

\subsection{Определения}
Степенной метод -- простейший метод решения частичных проблем собственных значений, основа для построения более эффективных методов.

Пусть $\mathbf{A}$ -- вещественная $n\times n$ матрица простой структуры, то есть имеет ровно $n$ линейно независимых собственных векторов:
\begin{equation}
    \mathbf{x}_{i=1,..,n} =
    \begin{pmatrix}
        x_{i1}\\ x_{i2} \\ \vdots \\ x_{in}
    \end{pmatrix},
\end{equation}
и пусть соответствующие им собственные значения расположены в порядке убывания:
\begin{equation}
    |\lambda_1| > |\lambda_2| \ge \ldots \ge |\lambda_n|,  ~~~~~ \text{1-е неравенство строгое.}
\end{equation}
Тогда задача степенного метода -- приближённое вычисление вещественного  $\lambda_1$ -- собственного числа с наибольшим модулем и соответствующего ему собственного вектора $\mathbf{x}_1$ матрицы  $\mathbf{A}$.

\subsection{Алгоритмы}
\subsubsection{Степенной метод для поиска максимального с.ч. без нормировки}
\begin{enumerate}
\item Ввести матрицу $\mathbf{A}$, задать произвольный вектор $\mathbf{\lambda}^{(0)}$ и произвольный ненулевой вектор $\mathbf{y}^{(0)}$ длиной $n$, задать допуск $\delta$, положить $k:=1$;

\item Выполнить итерацию: $\mathbf{y}^{(k)} = \mathbf{Ay}^{(k-1)}$;

\item Вычислить $\mathbf{\lambda}^{(k)} = \left( y^{(k)}_i / y^{(k-1)}_i \right)$ для элементов, у которых $ |y^{(k-1)}_i| > \delta$ -- поэлементное деление;

\item Если $|\langle\lambda^{(k)}\rangle-\langle\lambda^{(k-1)}\rangle| \le \delta ~~~ \implies ~~~
\tilde{\lambda_1} := \langle\lambda^{(k)}\rangle, ~~
\dfrac{\tilde{\mathbf{x}_1}}{||\tilde{\mathbf{x}_1}||} := \dfrac{\mathbf{y}^{(k)}}{||\mathbf{y}^{(k)}||} ~~~
\implies ~~~ \blacksquare$;

\item $k:= k+1$, перейти к 2;
\end{enumerate}

\paragraph{Комментарий:} С ростом k вектор $\mathbf{y}^{(k)}$ будет давать всё лучшее и лучшее приближение {\it к направлению} $\mathbf{x}_1$. Нетрудно заметить, что достаточно большом числе итераций k счет множителя  $\lambda^{(k)}$ в процессе счета может, в зависимости от модуля наибольшего с.ч., произойти либо превышение допустимых для используемого компьютера чисел, если $|\lambda_1| >1$, либо пропадание значащих цифр итерированных векторов, если $|\lambda_1| < 1$. Решению данной проблемы адресован

\subsubsection{Степенной метод для поиска максимального с.ч. c нормировкой}
\begin{enumerate}
    \item Ввести матрицу $\mathbf{A}$, задать произвольный вектор $\mathbf{\lambda}^{(0)}$ и произвольный ненулевой вектор $\mathbf{y}^{(0)}$ длиной $n$, вычислить $||\mathbf{y}^{(0)}||$ и вектор $\mathbf{x}^{(0)}:=~\dfrac{\mathbf{y}^{(0)}}{||\mathbf{y}^{(0)}||}$, задать допуск $\delta$, положить $k:=1$;

    \item Выполнить итерацию: $\mathbf{y}^{(k)} = \mathbf{Ax}^{(k-1)}$

    \item Вычислить $||\mathbf{y}^{(k)}||$ и $\mathbf{x}^{(k)} := \mathbf{y}^{(k)}/||\mathbf{y}^{(k)}||$

    \item Вычислить $\mathbf{\lambda}^{(k)} = \left( y^{(k)}_i / x^{(k-1)}_i \right)$ для элементов, у которых $ |x^{(k-1)}_i| > \delta$ -- поэлементное деление;

    \item Если $|\langle \lambda^{(k)}\rangle-\langle\lambda^{(k-1)}\rangle| \le \delta
    ~~~ \implies ~~~
    \tilde{\lambda_1} := \langle\lambda^{(k)} \rangle, ~~
    \dfrac{\tilde{\mathbf{x}_1}}{||\tilde{\mathbf{x}_1}||} := \mathbf{x}^{(k)} ~~~
    \implies ~~~ \blacksquare$

    \item $k:= k+1$, перейти к 2;
\end{enumerate}

\paragraph{Комментарий:} Описанный выше PM-алгортм с нормировкой, лишён недостатков наивного алгоритма связанных с техническими особенностями чисел с плавающей точкой, однако в остальном его стабильность столь же хрупка. Если на шаге 3 в ходе выполнения алгоритма достаточное количество раз в знаменателе будет оказываться число не превосходящее допуск $\delta$, значений $\lambda^{(k)}_i$ будет становиться всё меньше, пока они совсем не исчезнут. Также вызывает вопросы последний шаг алгоритма, где проверяется условие выхода. Этот шаг описан из рациональных соображений и не может гарантировать во всех случаях (даже при сделанных допущениях) получения собственной пары $\{\tilde{\lambda_1}, \tilde{\mathbf{x}_1}\}$ с наперёд заданной точностью, поскольку при разработке метода не было получено никаких оценок погрешности.

\subsubsection{Степенной метод для поиска минимального с.ч. c применением сдвига}
Поскольку для пары собственного вектора $\mathbf{x}$ и соответствующего ему собственного числа $\lambda$ верно
\begin{equation}
    \mathbf{Ax=\lambda x},
\end{equation}
то описанный выше степенной метод применим и для поиска наименьшего по модулю собственного числа $\lambda_n$ и соответствующего ему собственного вектора $\mathbf{x}_n$ знакоопределённой матрицы $\mathbf{A}$:
\begin{enumerate}
    \item Найти наибольшее по модулю собственное число $\lambda_1$ матрицы $\mathbf{A}$ используя алгоритм 1.2.2;
    \item Найти наибольшее по модулю собственное число $\Lambda$ матрицы $\mathbf{A-\lambda_1I}$ и соответствующий ему собственный вектор $\chi$ используя алгоритм 1.2.2;
    \item Искомая пара: $\{\lambda_n := \Lambda + \lambda_1, ~~ \mathbf{x}_n := \chi\}$
\end{enumerate}